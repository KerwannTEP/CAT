% Created 2021-01-13 mer. 22:12
% Intended LaTeX compiler: pdflatex
\documentclass[11pt]{article}
\usepackage[utf8]{inputenc}
\usepackage[T1]{fontenc}

\usepackage[a4paper,bindingoffset=0.2in,%
            left=0.5in,right=0.5in,top=0.5in,bottom=0.5in,%
            footskip=.25in]{geometry}

\usepackage{graphicx}
\usepackage{amsmath}
\usepackage{amssymb}
\usepackage{xspace}
\usepackage{commath}
\usepackage{times}
\usepackage{bm} 
\usepackage{balance}
\usepackage{hyperref}
\usepackage{mathtools}
\usepackage{stmaryrd}

%%%% Define Acronyms
\usepackage{acronym}
%
\newacro{NR}{non-resonant relaxation}
\newcommand{\NR}{\ac{NR}}
\newcommand{\rs}{\mathrm{s}}
\newcommand{\rt}{\mathrm{t}}
\newcommand{\rr}{\mathrm{r}}
\newcommand{\psis}{\psi_{\rs}}

\author{Kerwann}
\date{\today}
\title{Notes}
\hypersetup{
 pdfauthor={Kerwann},
 pdftitle={Notes},
 pdfkeywords={},
 pdfsubject={},
 pdfcreator={Emacs 27.1 (Org mode 9.3)}, 
 pdflang={English}}
\begin{document}

\maketitle

\section{Plummer model}
\label{sec:Plummer}

  Consider a Plummer model (Dejonghe, H.1987, MNRAS 224, 13) with potential
with units $r_{\rs}$ the Plummer scale radius (which sets the size
of the cluster core), $M$ the total mass of the cluster and $\bar{\tau}$
some unit time. Let $\psis$ be defined by

\begin{equation}
\psis = \frac{G M}{r_{\rs}} ,
\label{def_psi_s}
\end{equation}

for the central potential

\begin{equation}
\psi(r)=\frac{\psis}{\sqrt{1+r^{2}}} .
\label{def_potential}
\end{equation}

Let use fix $G=1\,r_{\rs}^{3}.M^{-1}.{\rm \bar{\tau}^{-2}}$ in the
new units so that $\psis=1\,r_{\rs}^{2}\cdot\bar{\tau}^{-2}$. This
fixes the time unit $\bar{\tau}$, as we have the relation. Therefore, in
those units the potential (per unit mass) is given by

\begin{equation}
\psi(r)=\frac{1}{\sqrt{1+r^{2}}} .
\label{def_potential_new_units}
\end{equation}

Define, given a radius $r$, the angular momentum $L(r,v_{\rr},v_{\rt})$
and binding energy per unit mass $E(r,v_{\rr},v_{\rt})$, functions of
the radial velocity $v_{\rr}$ and the tangential velocity $v_{\rt}\geq 0$
(defined as $\boldsymbol{v}=\boldsymbol{v_{\rr}}+\boldsymbol{v_{\rt}}=v_{\rr}\hat{\boldsymbol{r}}+\boldsymbol{v_{\rt}}$),
as

\begin{equation}
\begin{array}{ccl}
E(r,v_{\rr},v_{\rt}) & = & \psi(r)-\frac{1}{2}v_{\rr}^{2}-\frac{1}{2}v_{\rt}^{2} ,\\
L(r,v_{\rr},v_{\rt}) & = & r \cdot v_{\rt} ,
\end{array}
\label{eq:v_to_E_L}
\end{equation}
whose Jacobian is 
\begin{equation}
{\rm Jac}_{(r,v_{r},v_{t})\rightarrow(r,E,L)}=\left(\begin{array}{cc}
\frac{\partial E}{\partial v_{r}} & \frac{\partial E}{\partial v_{t}}\\
\frac{\partial L}{v_{ar}} & \frac{\partial L}{\partial v_{t}}
\end{array}\right)=\left(\begin{array}{cc}
-v_{r} & -v_{t}\\
0 & r
\end{array}\right)\Rightarrow|{\rm Jac}|=r|v_{r}| .
  \end{equation}







\end{document}
