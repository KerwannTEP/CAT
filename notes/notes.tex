% Created 2021-01-13 mer. 22:12
% Intended LaTeX compiler: pdflatex
\documentclass[11pt]{article}
\usepackage[utf8]{inputenc}
\usepackage[T1]{fontenc}

\usepackage[a4paper,bindingoffset=0.5in,%
            left=0.5in,right=0.5in,top=1in,bottom=1in,%
            footskip=.25in]{geometry}

\usepackage{graphicx}
\usepackage{amsmath}
\usepackage{amssymb}
\usepackage{xspace}
\usepackage{commath}
\usepackage{times}
\usepackage{bm} 
\usepackage{balance}
\usepackage{hyperref}
\usepackage{mathtools}
\usepackage{stmaryrd}

%%%% Define Acronyms
\usepackage{acronym}
%
\newacro{DF}{distribution fonction}
\newcommand{\DF}{\ac{DF}}
\newcommand{\rs}{\mathrm{s}}
\newcommand{\rt}{\mathrm{t}}
\newcommand{\rr}{\mathrm{r}}
\newcommand{\psis}{\psi_{\rs}}
\newcommand{\vr}{v_{\rr}}
\newcommand{\vt}{v_{\rt}}
\newcommand{\bv}{\boldsymbol{v}}
\newcommand{\bvt}{\boldsymbol{\vt}}
\newcommand{\hr}{\hat{\boldsymbol{r}}}
\newcommand{\Fq}{F_{q}}
\newcommand{\fq}{f_{q}}
\newcommand{\ra}{\mathrm{a}}
\newcommand{\va}{v_{\ra}}
\newcommand{\mH}{\mathbb{H}}
\newcommand{\HG}{\prescript{}{2}F_{1}}

\newcommand{\dvPar}{\Delta v_{||}}
\newcommand{\dvPerp}{\Delta v_{\perp}}

\newcommand{\dvParAvr}{\langle \dvPar \rangle_{\delta t}}
\newcommand{\dvParSqAvr}{\langle(\dvPar)^{2}\rangle_{\delta t}}
\newcommand{\dvPerpSqAvr}{\langle(\dvPerp)^{2}\rangle_{\delta t}}

\newcommand{\dvParAvrLoc}{\langle \dvPar \rangle}
\newcommand{\dvParSqAvrLoc}{\langle(\dvPar)^{2}\rangle}
\newcommand{\dvPerpSqAvrLoc}{\langle(\dvPerp)^{2}\rangle}

\newcommand{\bva}{\boldsymbol{\va}}
\newcommand{\ma}{m_{\ra}}

\newcommand{\e}[1]{\boldsymbol{e_{#1}}}
\newcommand{\bV}[1]{\boldsymbol{V_{#1}}}

\newcommand{\rd}{{\rm{d}}}
\newcommand{\var}{v_{\ra \rr}}
\newcommand{\vat}{v_{\ra \rt}}

\newcommand{\vp}{v_{+}}
\newcommand{\vm}{v_{-}}
\newcommand{\bvp}{\boldsymbol{\vp}}
\newcommand{\bvm}{\boldsymbol{\vm}}


\author{Kerwann}
\date{\today}
\title{Notes}
\hypersetup{
 pdfauthor={Kerwann},
 pdftitle={Notes},
 pdfkeywords={},
 pdfsubject={},
 pdfcreator={Emacs 27.1 (Org mode 9.3)}, 
 pdflang={English}}
\begin{document}

\maketitle

\section{Plummer model}
\label{sec:Plummer}

  Consider a Plummer model (Dejonghe, H.1987, MNRAS 224, 13) with potential
with units $r_{\rs}$ the Plummer scale radius (which sets the size
of the cluster core), $M$ the total mass of the cluster and $\bar{\tau}$
some unit time. Let $\psis$ be defined by
\begin{equation}
\psis = \frac{G M}{r_{\rs}} ,
\label{eq:def_psi_s}
\end{equation}

for the central potential
\begin{equation}
\psi(r)=\frac{\psis}{\sqrt{1+r^{2}}} .
\label{eq:def_potential}
\end{equation}

Let use fix $G=1\,r_{\rs}^{3}.M^{-1}.{\rm \bar{\tau}^{-2}}$ in the
new units so that $\psis=1\,r_{\rs}^{2}\cdot\bar{\tau}^{-2}$. This
fixes the time unit $\bar{\tau}$, as we have the relation. Therefore, in
those units the potential (per unit mass) is given by
\begin{equation}
\psi(r)=\frac{1}{\sqrt{1+r^{2}}} .
\label{eq:def_potential_new_units}
\end{equation}

Define, given a radius $r$, the angular momentum $L(r,\vr,\vt)$
and binding energy per unit mass $E(r,\vr,\vt)$, functions of
the radial velocity $\vr$ and the tangential velocity $\vt\geq 0$
(defined as $\bv=\vr \hr + \bvt$),
as
\begin{equation}
\begin{array}{ccl}
E(r,\vr,\vt) & = & \psi(r)-\frac{1}{2} \vr^{2}-\frac{1}{2}\vt^{2} ,\\
L(r,\vr,\vt) & = & r \cdot \vt ,
\end{array}
\label{eq:v_to_E_L}
\end{equation}

whose Jacobian is
\begin{equation}
{\rm Jac}_{(r,\vr,\vt)\rightarrow(r,E,L)}=\left(\begin{array}{cc}
\frac{\partial E}{\partial \vr} & \frac{\partial E}{\partial \vt}\\
\frac{\partial L}{\partial \vr} & \frac{\partial L}{\partial \vt}
\end{array}\right)=\left(\begin{array}{cc}
-\vr & -\vt\\
0 & r
\end{array}\right)\Rightarrow|{\rm Jac}|=r|\vr| .
\label{eq:Jacobian}
  \end{equation}

To obtain a bijective transformation, we must chose wether to chose
$\vr\ge 0$ or $\vr\leq 0$. A priori, this choice might have an
impact on the result, but we will should that the local and orbit-averaged
diffusion coefficients are not that. The coordinate system is spherical,
its origin being at the center of the globular cluster. Finally, we
consider the corresponding anisotropic distribution functions of the
field stars $\Fq(r,E,L)= \Fq(E,L)$ in $(E,L)$-space. Since $(E,L)$
and $(\vr,\vt)$ are linked, we can make use of the following
equalities (for the moment, $\vr$ is defined modulo the sign)
\begin{equation}
\begin{array}{ccl}
\Fq(E,L) &=&\fq(r,\vr(r,E,L),\vt(r,E,L)) ,\\
\fq(r,\vr,\vt)&=&F_{q}(E(r,\vr,\vt),L(r,\vr,\vt)) ,
\end{array}
\label{eq:conversion_DF}
\end{equation}

where $\fq$ is the \DF in the $(\vr,\vt)$ space and where $q$ is an anisotropy
parameter:
\begin{itemize}
\item $q\in]0,2]$: radially anisotropic
\item $q=0$: isotropic
\item $q\in]-\infty,0[$: tangentially anisotropic.
\end{itemize}

Note that the DF has spherical symmetry in position. Its expression for $E \geq 0, L \geq 0$
is (for $q \neq0$):
\begin{equation}
  \Fq(E,L)=\frac{3\Gamma(6-q)}{2(2\pi)^{5/2}\Gamma(q/2)}E^{7/2-q}\mH(0,\frac{q}{2},\frac{9}{2}-q,1;\frac{L^{2}}{2E})
  \label{eq:def_Fq}
\end{equation}
where
\begin{equation}
\mH(a,b,c,d;x)=\begin{cases}
\frac{\Gamma(a+b)}{\Gamma(c-a)\Gamma(a+d)}x^{a} \HG(a+b,1+a-c,a+d;x) & {\rm{if}} \quad x\leq1 ,\\
\frac{\Gamma(a+b)}{\Gamma(d-b)\Gamma(b+c)}x^{-b} \HG(a+b,1+b-d,b+c;\frac{1}{x}) &{\rm{if}} \quad x\geq1 ,
\end{cases}
\label{eq:def_H}
\end{equation}
which reduces in the isotropic case $(q=0)$ to
\begin{equation}
  F_{0}(E)=\frac{3}{7\pi^{3}}(2E)^{7/2} ,
  \label{eq:F_q=0}
\end{equation}

and in the extreme radially anisotropic $(q=2)$ to
\begin{equation}
  F_{2}(E)=\begin{cases}
\frac{6}{(2\pi)^{3}}(2E-L)^{3/2} & {\rm{if}} \quad 2 E\leq L^{2} ,\\
0 &{\rm{if}} \quad  2 E\geq L^{2} .
\end{cases}
  \label{eq:F_q=2}
\end{equation}

When $E \leq 0$ or $L \leq 0$ then $\Fq(E,L) = 0$.




\section{Determination of the local diffusion coefficients}
\label{sec:LocDiffCoeffs}

The local diffusion coefficients are the average velocity changes
per unit time. We are interested in computing

\begin{equation}
\begin{array}{ccl}
  \dvParAvrLoc(r,\vr,\vt) & = & \frac{ \dvParAvr(r,\vr,\vt)}{\delta t} ,\\
 \dvParSqAvrLoc(r,\vr,\vt) & = & \frac{ \dvParSqAvr(r,\vr,\vt)}{\delta t}  ,\\
\dvPerpSqAvrLoc(r,\vr,\vt) & = & \frac{ \dvPerpSqAvr(r,\vr,\vt)}{\delta t} ,
\end{array}
\label{eq:DiffCoeffLoc}
\end{equation}

where the subscript are relative to the relative velocity of test star (in the referential where the deflecting field star is still). Consider a test star at position $r$, mass $m$ and initial velocity
$\bv$ which interacts with a field star with impact parameter
$b$, mass $\ma$ and velocity  $\bva$, Binney et Tremaine
(2008, eq. (L.7) page 834) gives , with the convention (here, parallel
and perpendicular to relative velocity)
\begin{equation}
  \Delta\boldsymbol{v}=- \dvPar \e1'+\dvPerp(-\e2'\cos\phi+\e3'\sin\phi) ,
  \label{eq:deltaV}
\end{equation}

where $\e1' \parallel \bV0$ and $\phi$
is the angle between the plane of the relative orbit and $\e2'$,

\begin{equation}
\begin{array}{ccl}
  \dvPerp & =&\frac{2m_{a}V_{0}}{m+m_{a}}\frac{b/b_{90}}{1+b^{2}/b_{90}^{2}} ,\\
 \dvPar & = & \frac{2m_{a}V_{0}}{m+m_{a}}\frac{1}{1+b^{2}/b_{90}^{2}} ,
\end{array}
\label{eq:delta_v}
\end{equation}

where $\bV0= \bv-\bva$ and $b_{90}$ is the 90° deflection radius,
given by eq (L.8) 
\begin{equation}
  b_{90}=\frac{G(m+m_{a})}{V_{0}^{2}} .
  \label{eq:b90}
\end{equation}

Furthermore, after averaging over the equiprobable angles $\phi$
(test star can be on either ``side'' of the field star), we obtain

\begin{equation}
\begin{array}{ccl}
  \langle\Delta v_{i}\rangle_{\phi} & =&-\dvPar \langle\boldsymbol{e_{i}},\e1'\rangle ,\\
  {}&{}&{} \\
  \langle\Delta v_{i}\Delta v_{j}\rangle_{\phi} & = & (\Delta v_{\parallel})^{2}\langle\boldsymbol{e_{i}},\boldsymbol{e_{1}'}\rangle\langle\boldsymbol{e_{j}},\boldsymbol{e_{1}'}\rangle \\
  {}&{}&+\frac{1}{2}(\dvPerp)^{2}\left[\langle\boldsymbol{e_{i}},\e2'\rangle\langle\boldsymbol{e_{j}},\e2'\rangle+\langle\boldsymbol{e_{i}},\e3'\rangle\langle\boldsymbol{e_{j}},\e3'\rangle\right]
\end{array}
\label{eq:delta_v}
\end{equation}

where $(e_{1},e_{2},e_{3})$ is an fixed, arbitrary coordonnate system. Here, note that when considering a test star with energy and angular
momentum (per unit mass) $(E,L)$, using the choise $v_{r}\geq0$
or the choice $v_{r}\leq0$ has an impact on the local change of velocity
through $V_{0}$.

We sum the effects of all the encounter up. Number density of field
stars (at position $r$) within velocity space volume ${\rm d}^{3}\boldsymbol{v_{a}}$
is $\fq(r,\bva) \rd^{3}\bva$ (remember
that $\fq(r,\bva)= \fq(r,\var,\vat)$). The number
of encounters in a time $\delta t$ with impact parameters between
$b$ and $b+\rd b$ is just this density times the volume of an
annulus with inner radius $b$, outer radius $b+\rd b$, and length
$V_{0}\delta t$, that is (eq. L9) $2\pi b{\rm d}bV_{0}\delta tf_{a}(r,\boldsymbol{v_{a}}){\rm d}^{3}\boldsymbol{v_{a}}$.

We sum up over the velocities and the impact parameters. For the latter,
we consider impact parameters between $0$ and a cut-off $b_{{\rm max}}$,
traditionally given approximately by the radius of the subject star
orbit.

Recall that $\bV0 =\bv -\bva$.
Since we assume that $\Lambda$ is large, we do not make any significant
additional error by replacing the factor $V_{0}$ in $\Lambda$ by
some typical stellar speed $v_{{\rm typ}}$, that is,

\begin{equation}
  \Lambda=\frac{b_{{\rm max}}v_{{\rm typ}}^{2}}{G(m+m_{a})} .
  \label{Lambda}
\end{equation}

This yields (Binney \& Tremaine, eq. L14)

\begin{equation}
\begin{array}{ccl}
  \langle\Delta v_{i}\rangle & =& \displaystyle{-4\pi\frac{\ma}{m+\ma}
    \int{\rd}^{3}\bva V_{0}^{2}b_{90}^{2} \fq(r,\bva)\ln\Lambda\langle
    \boldsymbol{e_{i}},\e1'\rangle} ,\\
  {}&{}&{} \\
  \langle\Delta v_{i}\Delta v_{j}\rangle & = &\displaystyle{ 4\pi\left(
    \frac{\ma}{m+\ma}\right)^{2}\int{\rd}^{3}\bva V_{0}^{3}b_{90}^{2}f_{a}(r,
    \bva)\ln\Lambda  \left[\langle\boldsymbol{e_{i}},\e2'\rangle\langle
      \boldsymbol{e_{j}},\e2'\rangle+\langle\boldsymbol{e_{i}},\e3'\rangle
      \langle\boldsymbol{e_{j}},\e3'\rangle\right]}
\end{array}
\label{eq:delta_v_sum}
\end{equation}


where we defined the Coulomb parameter $\Lambda=b_{{\rm max}}/b_{90}$.
Remark that the scalar products depend on $\boldsymbol{v_{a}}$. Take
$\Lambda=\lambda N$ (Binney et Tremaine, page 581) with $N\sim10^{5}$
and $\lambda=0.059$ (Hamilton et al. (2018), eq. (B37)) for a globular
cluster.

Using (Binney \& Tremaine, eq. L17 and L18), we obtain
\begin{equation}
\begin{array}{ccl}
  \langle\Delta v_{i}\rangle(r,\bv) & =&\displaystyle{4\pi G^{2}\ma(m+\ma)\ln\Lambda\frac{\partial h}{\partial v_{i}}(r,\bv)} ,\\
  {}&{}&{} \\
  \langle\Delta v_{i}\Delta v_{j}\rangle(r,\bv) & = &\displaystyle{4\pi G^{2}{\ma}^{2}\ln\Lambda\frac{\partial^{2}g}{\partial v_{i}\partial v_{j}}(r,\bv)}
\end{array}
\label{eq:delta_v_Rosenbluth}
\end{equation}
where the Rosenbluth potentials are defined as (Binney \& Tremaine,
eq. L19)
\begin{equation}
\begin{array}{ccl}
  h(r,\bv) & =&\displaystyle{\int{\rd}^{3}\bva\frac{\fq(r,\bva)}{|\bv-\bva|}} ,\\
  {}&{}&{} \\
  g(r,\bv) & = &\displaystyle{\int{\rd}^{3}\bva \fq(r,\bva)|\bv-\bva|}
\end{array}
\label{eq:Rosenbluth}
\end{equation}

\subsection{Anisotropic case}
\label{subsec:Aniso}

Since this result is valid for any arbitrary coordinate system, we
can fix it to the one where $\e1=\hat{v}$ and $\e2$ is the projection
of $\hr$ onto the equatorial plane orthogonal to $\e1$. Then
we'll have the relations
\begin{equation}
\begin{array}{ccl}
  \dvParAvrLoc(r,\bv) & =&\displaystyle{\langle\Delta v_{1}\rangle(r,\bv)} ,\\
  {}&{}&{} \\
  \dvParSqAvrLoc(r,\bv) & = &\displaystyle{\langle(\Delta v_{1})^{2}\rangle(r,\bv)}\\
  {}&{}&{} \\
  \dvPerpSqAvrLoc(r,\bv) & = &\displaystyle{\langle(\Delta v_{2})^{2}\rangle(r,\bv)+\langle(\Delta v_{3})^{2}\rangle(r,\bv)}
\end{array}
\label{eq:TestStarDeflection}
\end{equation}
where the subscripts are relative of the velocity of the test star.

and a tedious by straightforward computation see appendix) yields

\begin{equation}
\begin{array}{ccl}
  \dvParAvrLoc(r,\bv) & =&\displaystyle{4\pi G^{2}\ma(m+\ma)\ln\Lambda(\frac{\vr}{v}\frac{\partial h}{\partial \vr}+\frac{\vt}{v}\frac{\partial h}{\partial \vt})} ,\\
  {}&{}&{} \\
  \dvParSqAvrLoc(r,\bv) & = &\displaystyle{4\pi G^{2}{\ma}^{2}\ln\Lambda\left(\frac{\vr^{2}}{v^{2}}\frac{\partial^{2}g}{\partial \vr^{2}}+\frac{2\vr\vt}{v^{2}}\frac{\partial^{2}g}{\partial \vt\partial \vr}+\left(\frac{\vt}{v}\right)^{2}\frac{\partial^{2}g}{\partial \vt^{2}}\right)}\\
  {}&{}&{} \\
  \dvPerpSqAvrLoc(r,\bv) & = &\displaystyle{4\pi G^{2}\ma^{2}\ln\Lambda\left(\left(\frac{\vt}{v}\right)^{2}\frac{\partial^{2}g}{\partial \vr^{2}}-\frac{2\vr\vt}{v^{2}}\frac{\partial^{2}g}{\partial \vt\ partial \vr}+\left(\frac{\vr}{v}\right)^{2}\frac{\partial^{2}g}{\partial \vt^{2}}+\frac{1}{\vt}\frac{\partial g}{\partial \vt}\right)}   
\end{array}
\label{eq:AnisoLocDiffCoefsdRdT}
\end{equation}
where $h(r,\bva)=h(r,\vr,\vt)$ and $g(r,\bv)=g(r,\vr,\vt)$.

Applying the change of variable $\bV0=\bv-\bva$
and using spherical coordinates with axis $(Oz)=\hr$
the unit radius vector (parallel or antiparallel to the radial component
of $\bv$ by definition) yields
\begin{equation}
\begin{array}{cclcl}
  h(r,\vr,\vt) & =&\displaystyle{\int{\rd}^{3}\bV0 \frac{\fq(r,\bv-\bV0)}{V_{0}}} &=& \displaystyle{\int_{0}^{\infty}{\rd}V_{0} V_{0}\int_{0}^{\pi}{\rd}\theta\sin\theta\int_{0}^{2\pi}{\rd}\phi \fq(r,\bv-\bV0)} ,\\
  {}&{}&{} \\
  g(r,\vr,\vt) & = &\displaystyle{\int{\rd}^{3}\bV0 \fq(r,\bv-\bV0)V_{0}} &=& \displaystyle{\int_{0}^{\infty}{\rd}V_{0} V_{0}^{3}\int_{0}^{\pi}{\rd}\theta\sin\theta\int_{0}^{2\pi}{\rd}\phi f_{a}(r,\bv-\bV0)}
\end{array}
\label{eq:Rosenbluth}
\end{equation}
where
\begin{equation}
  \fq(r,\bv-\bV0)=\fq(r,\var,\vat)=\Fq(E(r,\var,\vat),L(r,\var,\vat))
  \label{eq:DFa}
\end{equation}

with $E,L$ given by eq \eqref{eq:v_to_E_L}.

For a given convention $+$ or $-$ of the choice of $\vr$, and
given $(E,L)$ the parameters of the test star, obtain the vectors
$\bvp=(|\vr|,\bvt)$ and $\bvm=(-|\vr|,\bvt)$,
which are symmetric with respect to the tangent plane where $\bvt$
lives. In terms of spherical coordinates, we have that $\bvp=(v,\theta_{0},0)$
and $\bvm=(v,\pi-\theta_{0},0)$. Remember that the
integration over the velocities $\bV0=\bv-\bva$ of
the field stars cover the whole $\boldsymbol{V_{0}}$-space. Given
a velocity $\bV0$ corresponds bijectively a field star
velocity $\bva$. The overall integration will in fact not depend on the convention we used. The $E(r,\var,\vat)$ component depends on the sign of $\vr$ since
\begin{equation}
  E_{a}(r,V_{0},\theta,\phi)=\psi(r)-\frac{1}{2}\left[v^{2}+V_{0}^{2}-2V_{0}(\vr\cos\theta+\vt\sin\theta\cos\phi)\right]
  \label{eq:Ea}
\end{equation}

but $L(r,\var,\vat)$ does not. When doing the integration, we will evaluate the integrand at both arguments $(V_{0},\theta,\phi)$ and  $(V_{0},\pi-\theta,\phi)$, and their summed contribution doesn't depend on the convention choice. In the following, we decide to use $\vr\geq 0$.



\end{document}
