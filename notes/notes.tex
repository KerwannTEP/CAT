% Created 2021-01-13 mer. 22:12
% Intended LaTeX compiler: pdflatex
\documentclass[11pt]{article}
\usepackage[utf8]{inputenc}
\usepackage[T1]{fontenc}

\usepackage[a4paper,bindingoffset=0.2in,%
            left=0.5in,right=0.5in,top=0.5in,bottom=0.5in,%
            footskip=.25in]{geometry}

\usepackage{graphicx}
\usepackage{amsmath}
\usepackage{amssymb}
\usepackage{xspace}
\usepackage{commath}
\usepackage{times}
\usepackage{bm} 
\usepackage{balance}
\usepackage{hyperref}
\usepackage{mathtools}
\usepackage{stmaryrd}

%%%% Define Acronyms
\usepackage{acronym}
%
\newacro{DF}{distribution fonction}
\newcommand{\DF}{\ac{DF}}
\newcommand{\rs}{\mathrm{s}}
\newcommand{\rt}{\mathrm{t}}
\newcommand{\rr}{\mathrm{r}}
\newcommand{\psis}{\psi_{\rs}}
\newcommand{\vr}{v_{\rr}}
\newcommand{\vt}{v_{\rt}}
\newcommand{\bv}{\boldsymbol{v}}
\newcommand{\bvt}{\boldsymbol{\vt}}
\newcommand{\hr}{\hat{\boldsymbol{r}}}
\newcommand{\Fq}{F_{q}}
\newcommand{\fq}{f_{q}}
\newcommand{\ra}{\mathrm{a}}
\newcommand{\va}{v_{\ra}}
\newcommand{\mH}{\mathbb{H}}
\newcommand{\HG}{\prescript{}{2}F_{1}}

\author{Kerwann}
\date{\today}
\title{Notes}
\hypersetup{
 pdfauthor={Kerwann},
 pdftitle={Notes},
 pdfkeywords={},
 pdfsubject={},
 pdfcreator={Emacs 27.1 (Org mode 9.3)}, 
 pdflang={English}}
\begin{document}

\maketitle

\section{Plummer model}
\label{sec:Plummer}

  Consider a Plummer model (Dejonghe, H.1987, MNRAS 224, 13) with potential
with units $r_{\rs}$ the Plummer scale radius (which sets the size
of the cluster core), $M$ the total mass of the cluster and $\bar{\tau}$
some unit time. Let $\psis$ be defined by
\begin{equation}
\psis = \frac{G M}{r_{\rs}} ,
\label{eq:def_psi_s}
\end{equation}

for the central potential
\begin{equation}
\psi(r)=\frac{\psis}{\sqrt{1+r^{2}}} .
\label{eq:def_potential}
\end{equation}

Let use fix $G=1\,r_{\rs}^{3}.M^{-1}.{\rm \bar{\tau}^{-2}}$ in the
new units so that $\psis=1\,r_{\rs}^{2}\cdot\bar{\tau}^{-2}$. This
fixes the time unit $\bar{\tau}$, as we have the relation. Therefore, in
those units the potential (per unit mass) is given by
\begin{equation}
\psi(r)=\frac{1}{\sqrt{1+r^{2}}} .
\label{eq:def_potential_new_units}
\end{equation}

Define, given a radius $r$, the angular momentum $L(r,\vr,\vt)$
and binding energy per unit mass $E(r,\vr,\vt)$, functions of
the radial velocity $\vr$ and the tangential velocity $\vt\geq 0$
(defined as $\bv=\vr \hr + \bvt$),
as
\begin{equation}
\begin{array}{ccl}
E(r,\vr,\vt) & = & \psi(r)-\frac{1}{2} \vr^{2}-\frac{1}{2}\vt^{2} ,\\
L(r,\vr,\vt) & = & r \cdot \vt ,
\end{array}
\label{eq:v_to_E_L}
\end{equation}

whose Jacobian is
\begin{equation}
{\rm Jac}_{(r,\vr,\vt)\rightarrow(r,E,L)}=\left(\begin{array}{cc}
\frac{\partial E}{\partial \vr} & \frac{\partial E}{\partial \vt}\\
\frac{\partial L}{\partial \vr} & \frac{\partial L}{\partial \vt}
\end{array}\right)=\left(\begin{array}{cc}
-\vr & -\vt\\
0 & r
\end{array}\right)\Rightarrow|{\rm Jac}|=r|\vr| .
\label{eq:Jacobian}
  \end{equation}

To obtain a bijective transformation, we must chose wether to chose
$\vr\ge 0$ or $\vr\leq 0$. A priori, this choice might have an
impact on the result, but we will should that the local and orbit-averaged
diffusion coefficients are not that. The coordinate system is spherical,
its origin being at the center of the globular cluster. Finally, we
consider the corresponding anisotropic distribution functions of the
field stars $\Fq(r,E,L)= \Fq(E,L)$ in $(E,L)$-space. Since $(E,L)$
and $(\vr,\vt)$ are linked, we can make use of the following
equalities (for the moment, $\vr$ is defined modulo the sign)
\begin{equation}
\begin{array}{ccl}
\Fq(E,L) &=&\fq(r,\vr(r,E,L),\vt(r,E,L)) ,\\
\fq(r,\vr,\vt)&=&F_{q}(E(r,\vr,\vt),L(r,\vr,\vt)) ,
\end{array}
\label{eq:conversion_DF}
\end{equation}

where $\fq$ is the \DF in the $(\vr,\vt)$ space and where $q$ is an anisotropy
parameter:
\begin{itemize}
\item $q\in]0,2]$: radially anisotropic
\item $q=0$: isotropic
\item $q\in]-\infty,0[$: tangentially anisotropic.
\end{itemize}

Note that the DF has spherical symmetry in position. Its expression for $E \geq 0, L \geq 0$
is (for $q \neq0$):
\begin{equation}
  \Fq(E,L)=\frac{3\Gamma(6-q)}{2(2\pi)^{5/2}\Gamma(q/2)}E^{7/2-q}\mH(0,\frac{q}{2},\frac{9}{2}-q,1;\frac{L^{2}}{2E})
  \label{eq:def_Fq}
\end{equation}
where
\begin{equation}
\mH(a,b,c,d;x)=\begin{cases}
\frac{\Gamma(a+b)}{\Gamma(c-a)\Gamma(a+d)}x^{a} \HG(a+b,1+a-c,a+d;x) & {\rm{if}} \quad x\leq1 ,\\
\frac{\Gamma(a+b)}{\Gamma(d-b)\Gamma(b+c)}x^{-b} \HG(a+b,1+b-d,b+c;\frac{1}{x}) &{\rm{if}} \quad x\geq1 ,
\end{cases}
\label{eq:def_H}
\end{equation}
which reduces in the isotropic case $(q=0)$ to
\begin{equation}
  F_{0}(E)=\frac{3}{7\pi^{3}}(2E)^{7/2} ,
  \label{eq:F_q=0}
\end{equation}

and in the extreme radially anisotropic $(q=2)$ to
\begin{equation}
  F_{2}(E)=\begin{cases}
\frac{6}{(2\pi)^{3}}(2E-L)^{3/2} & {\rm{if}} \quad 2 E\leq L^{2} ,\\
0 &{\rm{if}} \quad  2 E\geq L^{2} .
\end{cases}
  \label{eq:F_q=2}
\end{equation}

When $E \leq 0$ or $L \leq 0$ then $\Fq(E,L) = 0$.

\section{Determination of the local diffusion coefficients}
\label{sec:LocDiffCoeffs}

The local diffusion coefficients are the average velocity changes
per unit time. We are interested in computing (bad notation: those
are relative to the test star velocity, as opposed to the relative
velocity!!!)

\end{document}
