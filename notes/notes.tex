% Created 2021-01-13 mer. 22:12
% Intended LaTeX compiler: pdflatex
\documentclass[11pt]{article}
\usepackage[utf8]{inputenc}
\usepackage[T1]{fontenc}

\usepackage{graphicx}
\usepackage{amsmath}
\usepackage{amssymb}
\usepackage{xspace}
\usepackage{commath}
\usepackage{times}
\usepackage{bm} 
\usepackage{balance}
\usepackage{hyperref}
\usepackage{mathtools}
\usepackage{stmaryrd}

%%%% Define Acronyms
\usepackage{acronym}
%
\newacro{NR}{non-resonant relaxation}
\newcommand{\NR}{\ac{NR}}
\newcommand{\rs}{\mathrm{s}}
\newcommand{\psis}{\psi_{\rs}}

\author{Kerwann}
\date{\today}
\title{Notes}
\hypersetup{
 pdfauthor={Kerwann},
 pdftitle={Notes},
 pdfkeywords={},
 pdfsubject={},
 pdfcreator={Emacs 27.1 (Org mode 9.3)}, 
 pdflang={English}}
\begin{document}

\maketitle

\section{Plummer model}
\label{sec:Plummer}

  Consider a Plummer model (Dejonghe, H.1987, MNRAS 224, 13) with potential
with units \(r_{s}\) the Plummer scale radius (which sets the size
of the cluster core), \(M\) the total mass of the cluster and \(\bar{\tau}\)
some unit time. Let $\psis$ be defined by

\begin{equation}
\psis = \frac{G M}{r_{s}} ,
\label{def_psi_s}
\end{equation}

for the central potential

\begin{equation}
\psi(r)=\frac{\psis}{\sqrt{1+r^{2}}} .
\label{def_potential}
\end{equation}
\end{document}
