%% LyX 2.3.4.2 created this file.  For more info, see http://www.lyx.org/.
%% Do not edit unless you really know what you are doing.
\documentclass[english]{article}
\usepackage[T1]{fontenc}
\usepackage[latin9]{inputenc}
\usepackage{geometry}
\geometry{verbose,tmargin=3cm,bmargin=3cm,lmargin=2cm,rmargin=2cm}
\usepackage{textcomp}
\usepackage{amsmath}
\usepackage{amssymb}
\usepackage{wasysym}

\makeatletter

%%%%%%%%%%%%%%%%%%%%%%%%%%%%%% LyX specific LaTeX commands.
%% Because html converters don't know tabularnewline
\providecommand{\tabularnewline}{\\}

\makeatother

\usepackage{babel}
\begin{document}

\section{Plummer model}

Consider a Plummer model (Dejonghe, H.1987, MNRAS 224, 13) with potential
with units $r_{s}$ the Plummer scale radius (which sets the size
of the cluster core), $M$ the total mass of the cluster and $\bar{\tau}$
some unit time. Let $\psi_{s}$ be defined by
\[
\psi_{s}=\frac{GM}{r_{s}},
\]
for the central potential
\[
\psi(r)=\frac{\psi_{s}}{\sqrt{1+r^{2}}}.
\]
Let use fix $G=1\,r_{s}^{3}.M^{-1}.{\rm \bar{\tau}^{-2}}$ in the
new units so that $\psi_{s}=1\,r_{s}^{2}\cdot\bar{\tau}^{-2}$. This
fixes the time unit $\bar{\tau}$, as we have the relation. 
\[
G=\tilde{G}{\rm m}^{3}.{\rm kg}^{-1}.{\rm s^{-2}}=\tilde{G}\frac{{\rm m}^{3}}{r_{s}^{3}}\frac{{\rm kg}^{-1}}{M^{-1}}\frac{s^{-2}}{\bar{\tau}{}^{-2}}r_{s}^{3}.M_{\odot}^{-1}.{\rm \tau^{-2}}=\bar{G}r_{s}^{3}.M^{-1}.{\rm \bar{\tau}^{-2}},
\]
where $\tilde{G}=6.67430\times10^{-11}$. Consequently we can deduce
from $1=\tilde{G}\frac{{\rm m}^{3}}{r_{s}^{3}}\frac{M}{{\rm kg}}\frac{\bar{\tau}^{2}}{s^{2}}$
that
\[
\frac{\bar{\tau}}{s}=\sqrt{\frac{1}{\tilde{G}}\frac{r_{s}^{3}}{{\rm m}^{3}}\frac{{\rm kg}}{M}}.
\]
Therefore, in those units the potential (per unit mass) is given by
\[
\psi(r)=\frac{1}{\sqrt{1+r^{2}}}.
\]
Furthermore, letting $\tilde{c}=299792458$ and $c=\tilde{c\,}{\rm m}.{\rm s}^{-1}$
be the light speed in vacuum, we can express it in the new units by
\[
c=\tilde{c}\frac{{\rm m}}{r_{s}}\frac{\bar{\tau}}{{\rm s}}\,r_{s}.\bar{\tau}^{-1}=\bar{c}\,r_{s}.\bar{\tau}^{-1};\quad\bar{c}=\tilde{c}\frac{{\rm m}}{r_{s}}\frac{\bar{\tau}}{{\rm s}}
\]

We use those units from now on. For example, M3 has total mass $M=4.5\cdot10^{5}M_{\odot}$
and radius $r_{s}=90\,{\rm ly}$, hence $\bar{\tau}=1.0167\cdot10^{14}\,{\rm s}=3.2218\,{\rm Myr}$.
In those units, and $c=36\cdot10^{3}\,r_{s}.\bar{\tau}^{-1}$.

Define, given a radius $r$, the angular momentum $L(r,v_{r},v_{t})$
and binding energy per unit mass $E(r,v_{r},v_{t})$, functions of
the radial velocity $v_{r}$ and the tangential velocity $v_{t}\geq0$
(defined as $\boldsymbol{v}=\boldsymbol{v_{r}}+\boldsymbol{v_{t}}=v_{r}\hat{\boldsymbol{r}}+\boldsymbol{v_{t}}$),
as
\[
E(r,v_{r},v_{t})=\psi(r)-\frac{1}{2}v_{r}^{2}-\frac{1}{2}v_{t}^{2}
\]
\[
L(r,v_{r},v_{t})=rv_{t}
\]
which is a transformation with Jacobian 
\[
{\rm Jac}_{(r,v_{r},v_{t})\rightarrow(r,E,L)}=\left(\begin{array}{cc}
\frac{\partial E}{\partial v_{r}} & \frac{\partial E}{\partial v_{t}}\\
\frac{\partial L}{v_{ar}} & \frac{\partial L}{\partial v_{t}}
\end{array}\right)=\left(\begin{array}{cc}
-v_{r} & -v_{t}\\
0 & r
\end{array}\right)\Rightarrow|{\rm Jac}|=r|v_{r}|
\]
To obtain a bijective transformation, we must chose wether to chose
$v_{r}\ge0$ or $v_{r}\leq0$. A priori, this choice might have an
impact on the result, but we will should that the local and orbit-averaged
diffusion coefficients are not that. The coordinate system is spherical,
its origin being at the center of the globular cluster. Finally, we
consider the corresponding anisotropic distribution functions of the
field stars $F_{q}(r,E,L)=F_{q}(E,L)$ in $(E,L)$-space. Since $(E,L)$
and $(v_{r},v_{t})$ are linked, we can make use of the following
equalities (for the moment, $v_{r}$ is determined modulo the sign)
\[
F_{q}(r,E,L)=f_{a}(v_{r}(r,E,L),v_{t}(r,E,L))
\]
and its converse
\[
f_{a}(r,v_{r},v_{t})=F_{q}(E(r,v_{r},v_{t}),L(r,v_{r},v_{t}))
\]
where $q$ is an anisotropy parameter:
\begin{itemize}
\item $q\in]0,2]$: radially anisotropic
\item $q=0$: isotropic
\item $q\in]-\infty,0[$: tangentially anisotropic.
\end{itemize}
Note that the DF has spherical symmetry in position. Its expression
is (for $q\neq0$):
\[
F_{q}(E,L)=\frac{3\Gamma(6-q)}{2(2\pi)^{5/2}\Gamma(q/2)}E^{7/2-q}\mathbb{H}(0,\frac{q}{2},\frac{9}{2}-q,1;\frac{L^{2}}{2E})
\]
\[
\mathbb{H}(a,b,c,d;x)=\begin{cases}
\frac{\Gamma(a+b)}{\Gamma(c-a)\Gamma(a+d)}x^{a}._{2}F_{1}(a+b,1+a-c,a+d;x) & x\leq1\\
\frac{\Gamma(a+b)}{\Gamma(d-b)\Gamma(b+c)}x^{-b}._{2}F_{1}(a+b,1+b-d,b+c;\frac{1}{x}) & x\geq1
\end{cases}
\]
which reduces in the isotropic case $(q=0)$ to:
\[
F(E)=\frac{3}{7\pi^{3}}(2E)^{7/2}
\]


\subsection{Determination of the local diffusion coefficients}

The local diffusion coefficients are the average velocity changes
per unit time. We are interested in computing (bad notation: those
are relative to the test star velocity, as opposed to the relative
velocity!!!)
\[
\langle\Delta v_{||}\rangle(r,v_{r},v_{t})=\frac{\langle\Delta v_{||}\rangle_{\delta t}}{\delta t}
\]
\[
\langle(\Delta v_{||})^{2}\rangle(r,v_{r},v_{t})=\frac{\langle(\Delta v_{||})^{2}\rangle_{\delta t}}{\delta t}
\]
\[
\langle(\Delta v_{\perp})^{2}\rangle(r,v_{r},v_{t})=\frac{\langle(\Delta v_{\perp})^{2}\rangle_{\delta t}}{\delta t}
\]
Consider a test star at position $r$, mass $m$ and initial velocity
$\boldsymbol{v}$ which interacts with a field star with impact parameter
$b$, mass $m_{a}$ and velocity $\boldsymbol{v}$, Binney et Tremaine
(2008, eq. (L.7) page 834) gives , with the convention (here, parallel
and perpudicular to relative velocity)
\[
\Delta\boldsymbol{v}=-\Delta v_{\parallel}\boldsymbol{e_{1}'}+\Delta v_{\perp}(-\boldsymbol{e_{2}'}\cos\phi+\boldsymbol{e_{3}'}\sin\phi),
\]
where $\boldsymbol{e_{1}'}\parallel\boldsymbol{V_{0}}$ and $\phi$
is the angle between the plane of the relative orbit and $\boldsymbol{e_{2}'}$,
\[
\Delta v_{\perp}=\frac{2m_{a}V_{0}}{m+m_{a}}\frac{b/b_{90}}{1+b^{2}/b_{90}^{2}}
\]
\[
\Delta v_{||}=\frac{2m_{a}V_{0}}{m+m_{a}}\frac{1}{1+b^{2}/b_{90}^{2}}
\]
where $\boldsymbol{V_{0}}=\boldsymbol{v}-\boldsymbol{v_{a}}$ and
$b_{90}$ is the 90� deflection radius, given by eq (L.8) 
\[
b_{90}=\frac{G(m+m_{a})}{V_{0}^{2}}
\]
Remember that in our units, $G=1$ and $m,m_{a}$ are given in fraction
of the total mass $M$ of the cluster.

Furthermore, after averaging over the equiprobable angles $\phi$
(test star can be on either ``side'' of the field star), we obtain
\[
\langle\Delta v_{i}\rangle_{\phi}=-\Delta v_{\parallel}\langle\boldsymbol{e_{i}},\boldsymbol{e_{1}'}\rangle
\]
\begin{align*}
\langle\Delta v_{i}\Delta v_{j}\rangle_{\phi} & =(\Delta v_{\parallel})^{2}\langle\boldsymbol{e_{i}},\boldsymbol{e_{1}'}\rangle\langle\boldsymbol{e_{j}},\boldsymbol{e_{1}'}\rangle\\
+ & \frac{1}{2}(\Delta v_{\perp})^{2}\left[\langle\boldsymbol{e_{i}},\boldsymbol{e_{2}'}\rangle\langle\boldsymbol{e_{j}},\boldsymbol{e_{2}'}\rangle+\langle\boldsymbol{e_{i}},\boldsymbol{e_{3}'}\rangle\langle\boldsymbol{e_{j}},\boldsymbol{e_{3}'}\rangle\right]
\end{align*}
where $(e_{1},e_{2},e_{3})$ is an fixed, arbitrary coordonnate system.

Here, note that when considering a test star with energy and angular
momentum (per unit mass) $(E,L)$, using the choise $v_{r}\geq0$
or the choice $v_{r}\leq0$ has an impact on the local change of velocity
through $V_{0}$. 

We sum the effects of all the encounter up. Number density of field
stars (at position $r$) within velocity space volume ${\rm d}^{3}\boldsymbol{v_{a}}$
is $f_{a}(r,\boldsymbol{v_{a}}){\rm d}^{3}\boldsymbol{v_{a}}$ (remember
that $f_{a}(r,\boldsymbol{v_{a}})=f_{a}(r,v_{ar},v_{at})$). The number
of encounters in a time $\delta t$ with impact parameters between
$b$ and $b+{\rm d}b$ is just this density times the volume of an
annulus with inner radius $b$, outer radius $b+{\rm d}b$, and length
$V_{0}\delta t$, that is (eq. L9)
\[
2\pi b{\rm d}bV_{0}\delta tf_{a}(r,\boldsymbol{v_{a}}){\rm d}^{3}\boldsymbol{v_{a}}
\]
We sum up over the velocities and the impact parameters. For the latter,
we consider impact parameters between $0$ and a cut-off $b_{{\rm max}}$,
traditionally given approximately by the radius of the subject star
orbit. 

Recall that $\boldsymbol{V_{0}}=\boldsymbol{v}-\boldsymbol{v_{a}}$.
Since we assume that $\Lambda$ is large, we do not make any significant
additional error by replacing the factor $V_{0}$ in $\Lambda$ by
some typical stellar speed $v_{{\rm typ}}$, that is, 
\[
\Lambda=\frac{b_{{\rm max}}v_{{\rm typ}}^{2}}{G(m+m_{a})}.
\]
 This yields (Binney \& Tremaine, eq. L14)
\[
\langle\Delta v_{i}\rangle(r,\boldsymbol{v})=-4\pi\frac{m_{a}}{m+m_{a}}\int{\rm d}^{3}\boldsymbol{v_{a}}V_{0}^{2}b_{90}^{2}f_{a}(r,\boldsymbol{v_{a}})\ln\Lambda\langle\boldsymbol{e_{i}},\boldsymbol{e_{1}'}\rangle
\]
\begin{align*}
\langle\Delta v_{i}\Delta v_{j}\rangle(r,\boldsymbol{v}) & =4\pi\left(\frac{m_{a}}{m+m_{a}}\right)^{2}\int{\rm d}^{3}\boldsymbol{v_{a}}V_{0}^{3}b_{90}^{2}f_{a}(r,\boldsymbol{v_{a}})\ln\Lambda\\
\times & \left[\langle\boldsymbol{e_{i}},\boldsymbol{e_{2}'}\rangle\langle\boldsymbol{e_{j}},\boldsymbol{e_{2}'}\rangle+\langle\boldsymbol{e_{i}},\boldsymbol{e_{3}'}\rangle\langle\boldsymbol{e_{j}},\boldsymbol{e_{3}'}\rangle\right]
\end{align*}
where we defined the Coulomb parameter $\Lambda=b_{{\rm max}}/b_{90}$.
Remark that the scalar products depend on $\boldsymbol{v_{a}}$. Take
$\Lambda=\lambda N$ (Binney et Tremaine, page 581) with $N\sim10^{5}$
and $\lambda=0.059$ (Hamilton et al. (2018), eq. (B37)) for a globular
cluster. 

Then, using (Binney \& Tremaine, eq. L17)
\[
\langle\boldsymbol{e_{i}},\boldsymbol{e_{1}'}\rangle=\frac{V_{0i}}{V_{0}};\quad\langle\boldsymbol{e_{i}},\boldsymbol{e_{2}'}\rangle\langle\boldsymbol{e_{j}},\boldsymbol{e_{2}'}\rangle+\langle\boldsymbol{e_{i}},\boldsymbol{e_{3}'}\rangle\langle\boldsymbol{e_{j}},\boldsymbol{e_{3}'}\rangle=\delta_{ij}-\frac{V_{0i}V_{0j}}{V_{0}^{2}}
\]
we obtain
\[
\langle\Delta v_{i}\rangle(r,\boldsymbol{v})=-4\pi G^{2}m_{a}(m+m_{a})\ln\Lambda\int{\rm d}^{3}\boldsymbol{v_{a}}f_{a}(r,\boldsymbol{v_{a}})\frac{V_{0i}}{V_{0}^{3}}
\]
\begin{align*}
\langle\Delta v_{i}\Delta v_{j}\rangle(r,\boldsymbol{v}) & =4\pi G^{2}m_{a}^{2}\ln\Lambda\int{\rm d}^{3}\boldsymbol{v_{a}}\frac{f_{a}(r,\boldsymbol{v_{a}})}{V_{0}}\left(\delta_{ij}-\frac{V_{0i}V_{0j}}{V_{0}^{2}}\right)
\end{align*}
which can be written as (Binney \& Tremaine, eq. L18)
\[
\langle\Delta v_{i}\rangle(r,\boldsymbol{v})=4\pi G^{2}m_{a}(m+m_{a})\ln\Lambda\frac{\partial h}{\partial v_{i}}(r,\boldsymbol{v})
\]
\[
\langle\Delta v_{i}\Delta v_{j}\rangle(r,\boldsymbol{v})=4\pi G^{2}m_{a}^{2}\ln\Lambda\frac{\partial^{2}g}{\partial v_{i}\partial v_{j}}(r,\boldsymbol{v})
\]
where the Rosenbluth potentials are defined as (Binney \& Tremaine,
eq. L19)
\[
h(r,\boldsymbol{v})=\int{\rm d}^{3}\boldsymbol{v_{a}}\frac{f_{a}(r,\boldsymbol{v_{a}})}{|\boldsymbol{v}-\boldsymbol{v_{a}}|}
\]
\[
g(r,\boldsymbol{v})=\int{\rm d}^{3}\boldsymbol{v_{a}}f_{a}(r,\boldsymbol{v_{a}})|\boldsymbol{v}-\boldsymbol{v_{a}}|
\]


\subsubsection{Anisotropic case}

Since this result is valid for any arbitrary coordinate system, we
can fix it to the one where $e_{1}=\hat{v}$ and $e_{2}$ is the projection
of $\hat{r}$ onto the equatorial plane orthogonal to $e_{1}$. Then
we'll have the relations
\[
\langle\Delta v_{||}\rangle=\langle\Delta v_{1}\rangle
\]
\[
\langle(\Delta v_{||})^{2}\rangle=\langle(\Delta v_{1})^{2}\rangle
\]
\[
\langle(\Delta v_{\perp})^{2}\rangle=\langle(\Delta v_{2})^{2}\rangle+\langle(\Delta v_{3})^{2}\rangle
\]
and a tedious by straightforward computation see appendix) yields
\[
\langle\Delta v_{||}\rangle=4\pi G^{2}m_{a}(m+m_{a})\ln\Lambda(\frac{v_{r}}{v}\frac{\partial h}{\partial v_{r}}+\frac{v_{t}}{v}\frac{\partial h}{\partial v_{t}})
\]
\[
\langle(\Delta v_{||})^{2}\rangle=4\pi G^{2}m_{a}^{2}\ln\Lambda\left(\frac{v_{r}^{2}}{v^{2}}\frac{\partial^{2}g}{\partial v_{r}^{2}}+\frac{2v_{r}v_{t}}{v^{2}}\frac{\partial^{2}g}{\partial v_{t}\partial v_{r}}+\left(\frac{v_{t}}{v}\right)^{2}\frac{\partial^{2}g}{\partial v_{t}^{2}}\right)
\]
\begin{align*}
\langle(\Delta v_{\perp})^{2}\rangle= & 4\pi G^{2}m_{a}^{2}\ln\Lambda\left(\left(\frac{v_{t}}{v}\right)^{2}\frac{\partial^{2}g}{\partial v_{r}^{2}}-\frac{2v_{r}v_{t}}{v^{2}}\frac{\partial^{2}g}{\partial v_{t}\partial v_{r}}+\left(\frac{v_{r}}{v}\right)^{2}\frac{\partial^{2}g}{\partial v_{t}^{2}}+\frac{1}{v_{t}}\frac{\partial g}{\partial v_{t}}\right)
\end{align*}
where $h(r,\boldsymbol{v})=h(r,v_{r},v_{t})$ and $g(r,\boldsymbol{v})=g(r,v_{r},v_{t})$.
Applying the change of variable $\boldsymbol{v'}=\boldsymbol{v}-\boldsymbol{v_{a}}=\boldsymbol{V_{0}}$
and using spherical coordinates with axis $(Oz)=\hat{\boldsymbol{r}}$
the unit radius vector (parallel or antiparallel to the radial component
of $\boldsymbol{v}$ by definition) yields
\[
h(r,v_{r},v_{t})=\int{\rm d}^{3}\boldsymbol{v'}\frac{f_{a}(r,\boldsymbol{v}-\boldsymbol{v'})}{v'}=\int_{0}^{\infty}{\rm d}v'v'\int_{0}^{\pi}{\rm d}\theta\sin\theta\int_{0}^{2\pi}{\rm d}\phi f_{a}(r,\boldsymbol{v}-\boldsymbol{v'})
\]
\[
g(r,v_{r},v_{t})=\int{\rm d}^{3}\boldsymbol{v'}f_{a}(r,\boldsymbol{v}-\boldsymbol{v'})v'=\int_{0}^{\infty}{\rm d}v'v'^{3}\int_{0}^{\pi}{\rm d}\theta\sin\theta\int_{0}^{2\pi}{\rm d}\phi f_{a}(r,\boldsymbol{v}-\boldsymbol{v'})
\]
where we have
\[
f_{a}(r,\boldsymbol{v}-\boldsymbol{v'})=f_{a}(r,v_{ar},v_{at})=F_{q}(E_{a}(r,v_{ar},v_{at}),L_{a}(r,v_{ar},v_{at}))
\]
with
\[
E_{a}(r,v_{ar},v_{at})=\psi(r)-\frac{1}{2}v_{ar}{}^{2}-\frac{1}{2}v_{at}{}^{2}
\]
\[
L_{a}(r,v_{ar},v_{at})=rv_{at}
\]

For a given convention $+$ or $-$ of the choice of $v_{r}$, and
given $(E,L)$ the parameters of the test star, obtain the vectors
$\boldsymbol{v_{+}}=(|v_{r}|,\boldsymbol{v_{t}})$ and $\boldsymbol{v_{-}}=(-|v_{r}|,\boldsymbol{v_{t}})$,
which are symmetric with respect to the tangent plane where $\boldsymbol{v_{t}}$
lives. In terms of spherical coordinates, we have that $\boldsymbol{v_{+}}=(v,\theta_{0},0)$
and $\boldsymbol{v_{-}}=(v,\pi-\theta_{0},0)$. Remember that the
integration over the velocities $\boldsymbol{v'}=\boldsymbol{v}-\boldsymbol{v_{a}}=\boldsymbol{V_{0}}$of
the field stars cover the whole $\boldsymbol{V_{0}}$-space. Given
a velocity $\boldsymbol{V_{0}}$ corresponds bijectively a field star
velocity $\boldsymbol{v_{a}}$.

We need to compute 
\[
h(r,\boldsymbol{v})=\int_{0}^{\infty}{\rm d}v'v'\int_{0}^{\pi}{\rm d}\theta\sin\theta\int_{0}^{2\pi}{\rm d}\phi f_{a}(r,\boldsymbol{v}-\boldsymbol{v'})
\]
\[
g(r,\boldsymbol{v})=\int_{0}^{\infty}{\rm d}v'v'^{3}\int_{0}^{\pi}{\rm d}\theta\sin\theta\int_{0}^{2\pi}{\rm d}\phi f_{a}(r,\boldsymbol{v}-\boldsymbol{v'})
\]
where we need to compute the radial and tangential components of the
vector $\boldsymbol{v_{a}}=\boldsymbol{v}-\boldsymbol{v'}$. This
is where we need to make sure that the choice of convention for $v_{r}$
doesn't change the overall result. A long but straighforward computation
gives the binding energy and the angular momentum at which the integrand
is evaluated:

\[
E_{a}(r,v',\theta,\phi)=\psi(r)-\frac{1}{2}\left[v^{2}+v'^{2}-2v'(v_{r}\cos\theta+v_{t}\sin\theta\cos\phi)\right]
\]
\[
L_{a}(r,v',\theta,\phi)=r(v_{t}^{2}+v'^{2}\sin^{2}\theta-2v_{t}v'\sin\theta\cos\phi)^{1/2}
\]
Now, remember that $\boldsymbol{v_{+}}=(|v_{r}|,\boldsymbol{v_{t}})$
and $\boldsymbol{v_{-}}=(-|v_{r}|,\boldsymbol{v_{t}})$ in the $+$
and $-$ convention. For a given $\boldsymbol{v'}=(v',\theta,\phi)$,
the angular momentum doesn't depend on the convention we chose since
only $v_{r}$ is impacted. On the other hand, for the bunding energy
per unit mass, we obtain
\[
E_{a+}(r,v',\theta,\phi)=\psi(r)-\frac{1}{2}\left[v^{2}+v'^{2}-2v'(|v_{r}|\cos\theta+v_{t}\sin\theta\cos\phi)\right]
\]
\[
E_{a-}(r,v',\theta,\phi)=\psi(r)-\frac{1}{2}\left[v^{2}+v'^{2}-2v'(-|v_{r}|\cos\theta+v_{t}\sin\theta\cos\phi)\right]
\]
which are different. However, when considering $\boldsymbol{\tilde{v}'}=(v',\pi-\theta,\phi)$,
which is another vector used in the integration, we have
\[
E_{a+}(r,v',\pi-\theta,\phi)=\psi(r)-\frac{1}{2}\left[v^{2}+v'^{2}-2v'(-|v_{r}|\cos\theta+v_{t}\sin\theta\cos\phi)\right]
\]
\[
E_{a-}(r,v',\pi-\theta,\phi)=\psi(r)-\frac{1}{2}\left[v^{2}+v'^{2}-2v'(|v_{r}|\cos\theta+v_{t}\sin\theta\cos\phi)\right]
\]
meaning that
\[
E_{a+}(r,v',\theta,\phi)=E_{a-}(r,v',\pi-\theta,\phi)
\]
\[
E_{a-}(r,v',\theta,\phi)=E_{a+}(r,v',\pi-\theta,\phi)
\]
Furthermore, the prefactor in the integrand $\sin\theta$ becomes
$\sin(\pi-\theta)=\sin\theta$ through this transformation. Therefore,
the convention doesn't change the result of the overall integration
(and therefore and the derivatives of the overall integration), and
we may chose to set $v_{r}\geq0$. For an actual computation, we also
need to compute the various velocity-partial derivatives of those
integrals, meaning that we need to compute the velocity-partial derivatives
of $f_{a}(r,\boldsymbol{v}-\boldsymbol{v'})=F_{q}(E_{a},L_{a})$ (exchange
derivation and integral). The calculation is done in the appendix,
and the results are (function are evaluated at $(E_{a},L_{a})$):

\[
\frac{\partial}{\partial v_{r}}\left[f_{a}(r,\boldsymbol{v}-\boldsymbol{v'})\right]=\left(-v_{r}+v'\cos\theta\right)\frac{\partial F}{\partial E}
\]
\[
\frac{\partial}{\partial v_{t}}\left[f_{a}(r,\boldsymbol{v}-\boldsymbol{v'})\right]=\left(-v_{t}+v'\sin\theta\cos\phi\right)\left(\frac{\partial F}{\partial E}-\frac{r}{L_{a}}\frac{\partial F}{\partial L}\right)
\]
\begin{align*}
\frac{\partial^{2}}{\partial v_{r}^{2}}\left[f_{a}(r,\boldsymbol{v}-\boldsymbol{v'})\right] & =-\frac{\partial F}{\partial E}+\left(-v_{r}+v'\cos\theta\right)^{2}\frac{\partial^{2}F}{\partial E{}^{2}}
\end{align*}
\begin{align*}
\frac{\partial^{2}}{\partial v_{t}\partial v_{r}}\left[f_{a}(r,\boldsymbol{v}-\boldsymbol{v'})\right] & =\left(-v_{r}+v'\cos\theta\right)\left(-v_{t}+v'\sin\theta\cos\phi\right)\left(\frac{\partial^{2}F}{\partial E{}^{2}}-\frac{r}{L_{a}}\frac{\partial^{2}F}{\partial L\partial E}\right)
\end{align*}
\begin{align*}
\frac{\partial^{2}}{\partial v_{t}^{2}}\left[f_{a}(r,\boldsymbol{v}-\boldsymbol{v'})\right] & =-\frac{\partial F}{\partial E}+\frac{r}{L_{a}}\frac{\partial F}{\partial L}+\left(-v_{t}+v'\sin\theta\cos\phi\right)^{2}\left(\frac{\partial^{2}F}{\partial E{}^{2}}-\frac{2r}{L_{a}}\frac{\partial^{2}F}{\partial L\partial E}-\frac{r^{2}}{L_{a}^{3}}\frac{\partial F}{\partial L}+\frac{r^{2}}{L_{a}^{2}}\frac{\partial^{2}F}{\partial L{}^{2}}\right)
\end{align*}

Let $\theta\in[0,\pi]$ and $\phi\in[0,2\pi]$, and consider
\[
Q(r,v',\theta,\phi)=v^{2}+v'^{2}-2v'(v_{r}\cos\theta+v_{t}\sin\theta\cos\phi)
\]
Then $Q(r,v',\theta,\phi)\geq v^{2}+v'^{2}-2v'(v_{t}+|v_{r}|)$ for
all $\theta,\phi$ in range, meaning that $Q(r,v',\theta,\phi)\rightarrow+\infty$
as $v'\rightarrow\infty$ uniformely in angles. Therefore, there exists
an bound $v'_{{\rm max}}$ such that 
\[
\forall v'>v'_{{\rm max}},\forall\theta\in[0,\pi],\forall\phi\in[0,2\pi];Q(r,v',\theta,\phi)>2\psi(r)
\]
that is,
\[
\forall v'>v'_{{\rm max}},\forall\theta\in[0,\pi],\forall\phi\in[0,2\pi];E_{a}(r,v',\theta,\phi)<0
\]
Above this bound, as the binding energy par unit mass is negative,
the DF of the field stars, evaluated for $(r,v',\theta,\phi)$, vanishes.
This shows that the $v'$-integral is in fact finite. We can obtain
this bound by solving the inequation in $v'$
\[
E_{a}(r,v',\theta,\phi)<0\Leftrightarrow v'^{2}-2v'(v_{t}+|v_{r}|)+v^{2}-2\psi(r)>0
\]
The $v'$ which satisfy this inequation are those which yields $E(r,v',\theta,\phi)<0$
(for any angle), hence a vanishing DF. Since the polynomial in $v'$
has non-negative leading coefficient (it is monic), the polynomial
is either always non-negative (negative discriminant) or there is
an closed interval over which it is negative (non-negative discriminant).
The polynomial's discriminant is 
\[
\Delta_{v}=4(v_{t}+|v_{r}|)^{2}-4(v^{2}-2\psi(r))=4(v_{t}^{2}+v_{r}^{2}+2|v_{r}|v_{t})-4(v^{2}-2\psi(r))=8(|v_{r}|v_{t}+\psi(r))
\]
Since $|v_{r}|,v_{t}\geq0$, it is non-negative. In that case, the
$v'$ over which we can integrate are those which are positive $(v'\geq0)$
and between the roots of the polynomial, given by

\[
v'_{\pm}=\frac{2(v_{t}+|v_{r}|)\pm\sqrt{8(|v_{r}|v_{t}+\psi(r))}}{2}=(v_{t}+|v_{r}|)\pm\sqrt{2(|v_{r}|v_{t}+\psi(r))}.
\]
One may ask if $v'_{-}$ is positive or negative. To that, recall
that $|v_{r}|=v\cos\theta_{0}$ and $v_{t}=v\sin\theta_{0}$ for $\theta_{0}\in[0,\pi/2]$.
Then 
\[
v'_{-}=v(\cos\theta+\sin\theta)-\sqrt{2(v^{2}\cos\theta\sin\theta+\psi(r))}
\]
We have that $v_{-}\leq0$ iif $v(\cos\theta_{0}+\sin\theta_{0})\leq\sqrt{2(v^{2}\cos\theta_{0}\sin\theta_{0}+\psi(r))}$.
Both sides are positive for $\theta_{0}\in[0,\pi/2]$, therefore $v_{-}\leq0$
iif 
\[
v^{2}(\cos\theta_{0}+\sin\theta_{0})^{2}\leq2(v^{2}\cos\theta_{0}\sin\theta_{0}+\psi(r))
\]
iff $v^{2}(1+2\cos\theta_{0}\sin\theta_{0})\leq2(v^{2}\cos\theta_{0}\sin\theta_{0}+\psi(r))$
iff $v^{2}\leq2\psi(r)$. Since $E=\psi(r)-\frac{1}{2}v^{2}$, this
is equivalent to $2(\psi(r)-E)\leq2\psi(r)$, i.e. $E\geq0$. Therefore,
since we study systems with $E\geq0$, it follows that we always have
$v_{-}\leq0$, meaning that our integration is over $[0,v'_{+}]$.
This upper bound depends on $E,L,r$ and is given by the formula as
long as $E\leq\psi(r)$ and $L\leq r\sqrt{2(\psi(r)-E)}$
\[
v_{+}=\left(\frac{L}{r}+\sqrt{2(\psi(r)-E)-\frac{L^{2}}{r^{2}}}\right)+\sqrt{2\left(\frac{L}{r}\sqrt{2(\psi(r)-E)-\frac{L^{2}}{r^{2}}}+\psi(r)\right)}
\]
This condition will always be satisfied in a star orbit (see next
section). It can be useful to define the effective binding potential
\[
\psi_{{\rm eff}}(r;L)=\psi(r)-\frac{L^{2}}{2r^{2}},
\]
so that we can rewrite the upper bound of the integral as 
\[
v_{{\rm max}}=\frac{L}{r}+\sqrt{2(\psi_{{\rm eff}}(r;L)-E)}+\sqrt{2\left(\frac{L}{r}\sqrt{2(\psi_{{\rm eff}}(r;L)-E)}+\psi(r)\right)}
\]
\[
K(r,v')=\int_{0}^{\pi}{\rm d}\theta\sin\theta\int_{0}^{2\pi}{\rm d}\phi f_{a}\left(r,v',\theta,\phi\right)
\]
where 
\[
f_{a}(r,v',\theta,\phi)=F_{q}(E_{a}(r,v',\theta,\phi),L_{a}(r,v',\theta,\phi))
\]
with
\[
E_{a}(r,v',\theta,\phi)=\psi(r)-\frac{1}{2}\left[v^{2}+v'^{2}-2v'(v_{r}\cos\theta+v_{t}\sin\theta\cos\phi)\right]
\]
\[
L_{a}(r,v',\theta,\phi)=r(v_{t}^{2}+v'^{2}\sin^{2}\theta-2v_{t}v'\sin\theta\cos\phi)^{1/2}
\]

If we want to use the Cuba.jl package with the Cuhre method (cuhre()
in Julia), we want to reduce our integral over an integration over
$[0,1]^{3}$. We must compute the two integrals
\[
h(r,\boldsymbol{v})=\int_{0}^{v_{\max}}{\rm d}v'v'\int_{0}^{\pi}{\rm d}\theta\sin\theta\int_{0}^{2\pi}{\rm d}\phi f_{a}(r,v',\theta,\phi)
\]
\[
g(r,\boldsymbol{v})=\int_{0}^{v_{\max}}{\rm d}v'v'^{3}\int_{0}^{\pi}{\rm d}\theta\sin\theta\int_{0}^{2\pi}{\rm d}\phi f_{a}(r,v',\theta,\phi)
\]
 Make the change of variable 
\[
\tilde{v'}=v'/v_{\max};\quad\tilde{\theta}=\theta/\pi;\quad\tilde{\phi}=\phi/2\pi.
\]
Then 
\[
h(r,\boldsymbol{v})=2\pi^{2}v_{\max}^{2}\int_{0}^{1}{\rm d}\tilde{v'}\tilde{v'}\int_{0}^{1}{\rm d}\tilde{\theta}\sin(\pi\tilde{\theta})\int_{0}^{1}{\rm d}\phi f_{a}(r,v_{\max}v',\pi\tilde{\theta},2\pi\tilde{\phi})
\]
\[
g(r,\boldsymbol{v})=2\pi^{2}v_{\max}^{4}\int_{0}^{1}{\rm d}\tilde{v'}\tilde{v'^{3}}\int_{0}^{1}{\rm d}\tilde{\theta}\sin(\pi\tilde{\theta})\int_{0}^{1}{\rm d}\phi f_{a}(r,v_{\max}v',\pi\tilde{\theta},2\pi\tilde{\phi})
\]
Note that the Cuba.jl Julia library calls the Cuba C library. The
cuhre function can be found:
\begin{itemize}
\item https://github.com/JohannesBuchner/cuba/blob/master/src/cuhre
\end{itemize}
May be interesting to find how to free memory to avoid getting millions
(or billions) of allocations. Maybe go to C?\\
\\
\\
~

Compute analytically? If not, finite differences?

\[
f'(x)\simeq\frac{f(x+\epsilon)-f(x-\epsilon)}{2\epsilon};\quad R_{1}(\epsilon)=\frac{f^{(3)}(\xi)}{6}\epsilon^{2},\xi\in(x-\epsilon,x+\epsilon)
\]
\[
f''(x)\simeq\frac{f(x+\epsilon)+f(x-\epsilon)-2f(x)}{\epsilon^{2}};\quad R_{2}(\epsilon)=\frac{f^{(4)}(\xi)}{12}\epsilon^{2},\xi\in(x-\epsilon,x+\epsilon)
\]


\subsubsection{Isotropic case}

We may want to check that the integrals yield the correct result.
To that end, it can be of interest to consider the simple case $q=0$,
where $f(E,L)=f(E)$, i.e. $f(r,\boldsymbol{v})=f(r,v)=F_{q}(E)$.
Then according Binney \& Tremaine, eq. (L26),

\[
\langle\Delta v_{\parallel}\rangle(r,\boldsymbol{v})=-\frac{16\pi^{2}G^{2}m_{a}(m+m_{a})\ln\Lambda}{v^{2}}\int_{0}^{v}{\rm d}v_{a}v_{a}^{2}f_{a}(v_{a})
\]
\[
\langle(\Delta v_{\parallel})^{2}\rangle(r,\boldsymbol{v})=\frac{32\pi^{2}G^{2}m_{a}^{2}\ln\Lambda}{3}\left[\int_{0}^{v}{\rm d}v_{a}\frac{v_{a}^{4}}{v^{3}}f_{a}(v_{a})+\int_{v}^{\infty}{\rm d}v_{a}v_{a}f_{a}(v_{a})\right]
\]
\[
\langle(\Delta v_{\perp})^{2}\rangle(r,\boldsymbol{v})=\frac{32\pi^{2}G^{2}m_{a}^{2}\ln\Lambda}{3}\left[\int_{0}^{v}{\rm d}v_{a}\left(\frac{3v_{a}^{2}}{v}-\frac{v_{a}^{4}}{v^{3}}\right)f_{a}(v_{a})+2\int_{v}^{\infty}{\rm d}v_{a}v_{a}f_{a}(v_{a})\right]
\]
Using the change of variable ${\rm d}E=-v{\rm d}v$ and that $F_{0}(E_{a})=0$
for $E_{a}<0$, this reads
\[
\langle\Delta v_{\parallel}\rangle(r,\boldsymbol{v})=-\frac{16\pi^{2}G^{2}m_{a}(m+m_{a})\ln\Lambda}{v^{2}}\int_{E}^{\psi(r)}{\rm d}E_{a}v_{a}F_{0}(E_{a})
\]
\[
\langle(\Delta v_{\parallel})^{2}\rangle(r,\boldsymbol{v})=\frac{32\pi^{2}G^{2}m_{a}^{2}\ln\Lambda}{3}\left[\int_{0}^{E}{\rm d}E_{a}F_{0}(E_{a})+\int_{E}^{\psi(r)}{\rm d}E_{a}\left(\frac{v_{a}}{v}\right)^{3}F_{0}(E_{a})\right]
\]
\[
\langle(\Delta v_{\perp})^{2}\rangle(r,\boldsymbol{v})=\frac{32\pi^{2}G^{2}m_{a}^{2}\ln\Lambda}{3}\left[2\int_{0}^{E}{\rm d}E_{a}F_{0}(E_{a})+\int_{E}^{\psi(r)}{\rm d}E_{a}\left(\frac{3v_{a}}{v}-\left(\frac{v_{a}}{v}\right)^{3}\right)F_{0}(E_{a})\right]
\]
where 
\[
v_{a}=v_{a}(r,E_{a})=\sqrt{2(\psi(r)-E_{a})}
\]
\[
F_{0}(E_{a})=\frac{3}{7\pi^{3}}(2E_{a})^{7/2}
\]
We can obviously compute the $[0,E]$ integrals
\[
K_{0}=\int_{0}^{E}{\rm d}E_{a}F_{0}(E_{a})=\frac{3}{7\pi^{3}}2{}^{7/2}\left[\frac{E_{a}{}^{9/2}}{9/2}\right]_{0}^{E}=\frac{1}{21\pi^{3}}(2E){}^{9/2}
\]
We only have to compute
\[
K_{1}=\int_{E}^{\psi(r)}{\rm d}E_{a}v_{a}F_{0}(E_{a})
\]
\[
K_{3}=\int_{E}^{\psi(r)}{\rm d}E_{a}v_{a}^{3}F_{0}(E_{a})
\]
Then
\[
\langle\Delta v_{\parallel}\rangle(r,\boldsymbol{v})=-\frac{16\pi^{2}G^{2}m_{a}(m+m_{a})\ln\Lambda}{v^{2}}K_{1}
\]
\[
\langle(\Delta v_{\parallel})^{2}\rangle(r,\boldsymbol{v})=\frac{32\pi^{2}G^{2}m_{a}^{2}\ln\Lambda}{3}\left[K_{0}+\frac{1}{v^{3}}K_{3}\right]
\]
\[
\langle(\Delta v_{\perp})^{2}\rangle(r,\boldsymbol{v})=\frac{32\pi^{2}G^{2}m_{a}^{2}\ln\Lambda}{3}\left[2K_{0}+\frac{3}{v}K_{1}-\frac{1}{v^{3}}K_{3}\right]
\]

In the appendix, we recompute the formulae of the isotropic case from
the arbitrary anisotropic case, with $v^{2}=v_{r}^{2}+v_{t}^{2}$,
$h(r,v_{r},v_{t})=h(r,v)$ and $g(r,v_{r},v_{t})=g(r,v)$.

\subsubsection{Local orbital parameter changes}

Now, switch to $(E,L)$ space and using eq. (C15) to (C19) of Bar-Or
\& Alexander (2016), which doesn't rely on an isotropy assumption,
we obtain (evaluate at $(r,\boldsymbol{v}(r,E,L))$) at first order
in $\Delta v/v$:
\[
\langle\Delta E\rangle(r,E,L)=-\frac{1}{2}\langle(\Delta v_{||})^{2}\rangle-\frac{1}{2}\langle(\Delta v_{\perp})^{2}\rangle-v\langle\Delta v_{||}\rangle
\]
\[
\langle(\Delta E)^{2}\rangle(r,E,L)=v^{2}\langle(\Delta v_{||})^{2}\rangle
\]
\[
\langle\Delta L\rangle(r,E,L)=\frac{L}{v}\langle\Delta v_{||}\rangle+\frac{r^{2}}{4L}\langle(\Delta v_{\perp})^{2}\rangle
\]
\[
\langle(\Delta L)^{2}\rangle(r,E,L)=\frac{L^{2}}{v^{2}}\langle(\Delta v_{||})^{2}\rangle+\frac{1}{2}\left(r^{2}-\frac{L^{2}}{v^{2}}\right)\langle(\Delta v_{\perp})^{2}\rangle
\]
\[
\langle\Delta E\Delta L\rangle(r,E,L)=-L\langle(\Delta v_{||})^{2}\rangle
\]
where
\[
r^{2}-\frac{L^{2}}{v^{2}}=\frac{r^{2}}{v^{2}}\left(v^{2}-\frac{L^{2}}{r^{2}}\right)=\frac{r^{2}}{v^{2}}\left(v^{2}-v_{t}^{2}\right)=\frac{r^{2}v_{r}^{2}}{v^{2}}
\]
Finally, due to our analysis, those quantities are well defined and
we can use the bijective transformation $(r,E,L)\leftrightarrow(r,v_{r},v_{t})$,
, through the relations
\[
E(r,v_{r},v_{t})=\psi(r)-\frac{1}{2}v_{r}{}^{2}-\frac{1}{2}v_{t}{}^{2}
\]
\[
L(r,v_{r},v_{t})=rv_{t}
\]
where $v_{r},v_{t}\geq0$, yielding $E\in]-\infty,\psi(r)]$ and $L\in[0,r\sqrt{2(\psi(r)-E)}]$,
while the converse relations are
\[
v_{t}(r,E,L)=\frac{L}{r}
\]
\[
v_{r}(r,E,L)=\sqrt{2(\psi_{{\rm eff}}(r;L)-E)}
\]
where $E\in]-\infty,\psi(r)]$ and $L\in[0,r\sqrt{2(\psi(r)-E)}]$,
yielding $v_{r},v_{t}\geq0$.

\subsection{Orbit of a test star in a globular cluster}

We can now compute the local diffusion coefficients $\langle\Delta E\rangle(r,E,L)$,
$\langle(\Delta E)^{2}\rangle(r,E,L)$, $\langle\Delta L\rangle(r,E,L)$,
$\langle(\Delta L)^{2}\rangle(r,E,L)$ and $\langle\Delta E\Delta L\rangle(r,E,L)$.
Since we are interested in the secular evolution of the system, we
can average over the dynamical time and smear out the star along its
orbit. This leads us to consider the orbit-average diffusion coefficients
\[
\overline{D_{X}}(E,L)\doteqdot\langle D_{X}\rangle_{\leftturn}(E,L)=\frac{1}{T}\int_{0}^{T}\langle\Delta X\rangle(r(t),E,L){\rm d}t=\frac{2}{T}\int_{r_{min}}^{r_{max}}\langle\Delta X\rangle(r,E,L)\frac{{\rm d}r}{v_{r}(r,E,L)}
\]
\[
\overline{D_{XY}}(E,L)\doteqdot\langle D_{XY}\rangle_{\leftturn}(E,L)=\frac{1}{T}\int_{0}^{T}\langle\Delta X\Delta Y\rangle(r(t),E,L){\rm d}t=\frac{2}{T}\int_{r_{min}}^{r_{max}}\langle\Delta X\Delta Y\rangle(r,E,L)\frac{{\rm d}r}{v_{r}(r,E,L)}
\]
where $v_{r}(r)$ is the radial velocity of the orbiting star at $r$.
However, this suppose that the star follow ``nice'' trajectories.
This is what we will look into in this section. Furthermore, since
$E,L$ are chosen so that the trajectory is a circular orbit with
radius $R$, then we apply the ``time'' formula instead of the radial
velocity one and obtain 
\[
\overline{D_{X}}(E,L)=\frac{1}{T}\int_{0}^{T}\langle\Delta X\rangle(R,E,L){\rm d}t=\langle\Delta X\rangle(R,E,L)
\]
\[
\overline{D_{XY}}(E,L)=\frac{1}{T}\int_{0}^{T}\langle\Delta X\Delta Y\rangle(R,E,L){\rm d}t=\langle\Delta X\Delta Y\rangle(R,E,L)
\]


\subsubsection{Study of an orbit}

-> See Kurth (1955), Astronomische Nachrichten, volume 282, Issue
6, p.241.

~

Consider a test star described by its position vector $\boldsymbol{r}$,
its binding energye (opposite of its energy) $E(t)$ and its angular
momentum vector $\boldsymbol{L}(t)$, per unit mass. Then
\[
E(t)=\psi(r(t))-\frac{1}{2}\boldsymbol{\dot{r}}^{2}(t)=\psi(r(t))-\frac{1}{2}\dot{r}^{2}(t)-\frac{1}{2}r^{2}(t)\dot{\theta}^{2}(t)
\]
\[
\boldsymbol{L}(t)=\boldsymbol{r}(t)\times\boldsymbol{v}(t)
\]
Differentiating the binding energy and using Newton's law shows that
it is conserved. Let $E(t)=E$. On the other hand, differentiating
$\boldsymbol{L}(t)$ and using the fact that the potential is central
shows that this quantity is also conserved. Therefore the star's orbit
is kept within a fixed plane determined by its initial conditions.
Let $L(t)=L$ be its conserved norm. We also have
\[
L=r(t)v_{t}(t)=r(t)^{2}\dot{\theta}(t)
\]
In our case, consider an orbit with binding energy $E\geq0$ and angular
momentum $L\geq0$. We can rewriting the energy conservation equation
(on one orbit) as
\[
E=\psi(r)-\frac{1}{2}\dot{r}^{2}-\frac{L^{2}}{2r^{2}}\Leftrightarrow\dot{r}^{2}=2(\psi(r)-E)-\frac{L^{2}}{r^{2}}=2(\psi_{{\rm eff}}(r;L)-E)
\]
Define
\[
v_{r}(r)\doteqdot\sqrt{2(\psi_{{\rm eff}}(r;L)-E)}\geq0
\]
Consider starting the motion from a radius with position initial radial
velocity. Then as long as the radial velocity is positive, we have
\[
\int_{0}^{t}\frac{\dot{r}(t){\rm d}t}{v_{r}(r(t))}=t\Leftrightarrow\int_{r_{{\rm min}}}^{r(t)}\frac{{\rm d}r}{v_{r}(r)}=t.
\]
This motion goes on until $\dot{r}(\tau)=0$ for some time $\tau$
defined by 
\[
\tau=\int_{r_{{\rm min}}}^{r_{\max}}\frac{{\rm d}r}{v_{r}(r)}
\]
where $r_{\max}$ is the radius reached at $\tau$. Then the $\ddot{r}$
(negative) decreases $r$ until a radius $r_{{\rm min}}$ which has
vanishing radial velocity (but positive $\ddot{r}$), and the process
repeats itself (see next section for a proof of those accelerations).
Note that the motion from $r_{\max}$ to $r_{\min}$ is symmetrical
to that from $r_{\min}$to $r_{\max}$, as it follows the relation
\[
\int_{\tau}^{t}\frac{\dot{r}(t){\rm d}t}{-v_{r}(r(t))}=t-\tau\Leftrightarrow\int_{r_{{\rm max}}}^{r(t)}\frac{{\rm d}r}{v_{r}(r)}=\tau-t
\]
hence
\[
\int_{r_{{\rm min}}}^{r(t)}\frac{{\rm d}r}{v_{r}(r)}-\int_{r_{{\rm min}}}^{r_{\max}}\frac{{\rm d}r}{v_{r}(r)}=\tau-t\Leftrightarrow\int_{r_{{\rm min}}}^{r(t)}\frac{{\rm d}r}{v_{r}(r)}=2\tau-t.
\]
Letting the orbit start at $r_{\min}$ and setting 
\[
F(r)=\int_{r_{\min}}^{r}\frac{dx}{v_{r}(x)},
\]
we obtain an expression for the radius of the orbit
\[
r(t)=\begin{cases}
F^{-1}(t) & t\in[0,\tau]\\
F^{-1}(2\tau-t) & t\in[\tau,2\tau]
\end{cases},
\]
with the symmetry $r(t)=r(2\tau-t)$ and the $2\tau$-periodicity
of the radius. Furthermore, notice that $1/v_{r}(r)$ is integrable
at $r_{\min}$ and $r_{\max}$. To show this, consider what happens
near $r_{\max}$. Let $\epsilon>0$.
\[
v_{r}^{2}(r_{\max}-\epsilon)=2(\psi(r_{\max}-\epsilon)-E)-\frac{L^{2}}{(r_{\max}-\epsilon)^{2}}=2(\psi(r_{\max})-\psi'(r_{\max})\epsilon+o(\epsilon)-E)-\frac{L^{2}}{r_{\max}^{2}}\frac{1}{(1-\frac{\epsilon}{r_{\max}})^{2}}
\]
\[
v_{r}^{2}(r_{\max}-\epsilon)=2(\psi(r_{\max})-\psi'(r_{\max})\epsilon+o(\epsilon)-E)-\frac{L^{2}}{r_{\max}^{2}}(1+\frac{2\epsilon}{r_{\max}}+o(\epsilon))
\]
\[
v_{r}^{2}(r_{\max}-\epsilon)=\underbrace{v_{r}^{2}(r_{\max})}_{=0}-2\psi'(r_{\max})\epsilon-\frac{2L^{2}\epsilon}{r_{\max}^{3}}+o(\epsilon)
\]
\[
v_{r}^{2}(r_{\max}-\epsilon)=\frac{2r_{\max}}{(1+r_{\max}^{2})^{3/2}}\epsilon-\frac{2L^{2}\epsilon}{r_{\max}^{3}}+o(\epsilon)
\]
Thus
\[
\frac{1}{v_{r}(r_{\max}-\epsilon)}=\left(\underbrace{\frac{2r_{\max}}{(1+r_{\max}^{2})^{3/2}}-\frac{2L^{2}}{r_{\max}^{3}}}_{=-2\psi_{{\rm eff}}'(r_{\max})\geq0}\right)^{-1/2}\frac{1}{\sqrt{\epsilon}}(1+o(1))
\]
A similar calculation yields
\[
\frac{1}{v_{r}(r_{\min}+\epsilon)}=\left(\underbrace{-\frac{2r_{\min}}{(1+r_{\min}^{2})^{3/2}}+\frac{2L^{2}}{r_{\min}^{3}}}_{=2\psi_{{\rm eff}}'(r_{\min})\geq0}\right)^{-1/2}\frac{1}{\sqrt{\epsilon}}(1+o(1))
\]
which is integrable since $1/\sqrt{\epsilon}$ is integrable at $0^{+}$
if the $()^{-1/2}$ term in strictly positive. Once we have shown
this, we can conclude that the orbit is ``rosette-like'', with periodical
radius of periode $T=2\tau$ with
\[
\tau=\int_{r_{{\rm min}}}^{r_{\max}}\frac{{\rm d}r}{v_{r}(r)}.
\]
Maybe isolate the borders and use change of variable $\psi_{{\rm eff}}(r;L)-E=\sin^{2}(\theta)$.
Then 
\[
\int_{r_{\min}}\frac{{\rm d}r}{v_{r}(r)}=\int_{0}\frac{2\cos\theta\sin\theta{\rm d}\theta}{\psi_{{\rm eff}}'(r;L)\sqrt{2\sin^{2}\theta}}=\int_{0}\frac{\sqrt{2}\cos\theta{\rm d}\theta}{\psi_{{\rm eff}}'(r;L)}
\]
with $\psi_{{\rm eff}}(r;L)-E=\sin^{2}(\theta)$. This change of variable
is posible because $\psi_{{\rm eff}}(r;L)\geq E$ and $E>0$, hence
$\psi_{{\rm eff}}(r;L)-E\geq0$ and $\psi_{{\rm eff}}(r;L)-E\leq\psi_{{\rm eff}}(r;L)\leq1$.
This transformation is bijective on $r\in[r_{\min},r_{*}^{L}]$ and
$r\in[r_{*}^{L},r_{\max}]$ where $\psi_{{\rm eff}}'(r_{*}^{L};L)=0$.
This way we have
\[
\frac{1}{\sqrt{1+r^{2}}}=\frac{L^{2}}{2r^{2}}+E+\sin^{2}\theta=\frac{L^{2}+2r^{2}(E+\sin^{2}\theta)}{2r^{2}}
\]
\[
\frac{1}{1+r^{2}}=\frac{(L^{2}+2r^{2}(E+\sin^{2}\theta))^{2}}{4r^{4}}
\]
Let $X=r^{2}$. Then 
\[
4X^{2}=(1+X)(L^{2}+2X(E+\sin^{2}\theta))^{2},
\]
degree-3 equation hence with the roots $X$ with an analytical expression.
Two solution between $r_{\min}$ and $r_{\max}$, each separated by
$r_{*}^{L}$. Therefore, letting $r_{\min}<r_{1}<r_{*}^{L}$ and $r_{*}^{L}<r_{2}<r_{\max}$,
decompose the integral into
\[
\int_{r_{{\rm min}}}^{r_{\max}}\frac{{\rm d}r}{v_{r}(r)}=\int_{r_{{\rm min}}}^{r_{1}}\frac{{\rm d}r}{v_{r}(r)}+\int_{r_{1}}^{r_{*}^{L}}\frac{{\rm d}r}{v_{r}(r)}+\int_{r_{*}^{L}}^{r_{2}}\frac{{\rm d}r}{v_{r}(r)}+\int_{r_{{\rm min}}}^{r_{\max}}\frac{{\rm d}r}{v_{r}(r)}
\]
\[
\int_{r_{{\rm min}}}^{r_{\max}}\frac{{\rm d}r}{v_{r}(r)}=\int_{0}^{\theta_{1}}\frac{\sqrt{2}\cos\theta{\rm d}\theta}{\psi_{{\rm eff}}'(r(\theta);L)}+\int_{r_{1}}^{r_{*}^{L}}\frac{{\rm d}r}{v_{r}(r)}+\int_{r_{*}^{L}}^{r_{2}}\frac{{\rm d}r}{v_{r}(r)}+\int_{0}^{\theta_{2}}\frac{\sqrt{2}\cos\theta{\rm d}\theta}{|\psi_{{\rm eff}}'(r(\theta);L)|}
\]
where $\psi_{{\rm eff}}(r_{1};L)-E=\sin^{2}(\theta_{1})$ and $\psi_{{\rm eff}}(r_{2};L)-E=\sin^{2}(\theta_{2})$. 

As for the angle, its derivative $\dot{\theta}$ is $T$-periodical
since $L=r^{2}\dot{\theta}$ with $r\geq0$ $T$-periodical. Therefore
it can be decomposed as 
\[
\theta(t)=\omega t+p(t)
\]
where $p(t)$ is $T$-periodical and $\omega$ is a real constant.
Indeed, let $\omega=\frac{1}{T}\int_{0}^{T}\dot{\theta}(t){\rm d}t$
and $p(t)=\theta(t)-\omega t$. Then 
\[
p(t+T)=\theta(t+T)-\omega(t+T)=\int_{0}^{t}\dot{\theta}(t){\rm d}t+\int_{t}^{t+T}\dot{\theta}(t){\rm d}t-\omega t-\omega T
\]
\[
p(t+T)=\left(\int_{0}^{t}\dot{\theta}(t){\rm d}t-\omega t\right)+\left(\int_{0}^{T}\dot{\theta}(t){\rm d}t-\int_{0}^{T}\dot{\theta}(t){\rm d}t\right)=(\theta(t)-\omega t)=p(t)
\]
showing that $p(t)$ is $T$-periodical. 

Now, to find what $r_{{\rm max}}$ and $r_{\min}$ are, we need to
solve $v_{r}(r)=0$, i.e. $\psi(r)-E-\frac{L^{2}}{2r^{2}}=0$, i.e.
\[
\frac{1}{\sqrt{1+r^{2}}}=\frac{L^{2}}{2r^{2}}+E=\frac{2Er^{2}+L^{2}}{2r^{2}}\Leftrightarrow\frac{1}{1+r^{2}}=\frac{(2Er^{2}+L^{2})^{2}}{4r^{4}}\Leftrightarrow4r^{4}=(2Er^{2}+L^{2})^{2}(1+r^{2})
\]
Let $X=r^{2}$. Then
\[
4X^{2}=(2EX+L^{2})^{2}(1+X)\Leftrightarrow4E^{2}X^{3}+4(E^{2}-1+EL^{2})X^{2}+(4EL^{2}+L^{4})X+L^{4}=0
\]
This is a degree-3 polynomial in $X$. It has 3 real roots iff its
discriminant $\Delta$ is strictly positive (two real roots, one of
which is double, if $\Delta=0$). Let
\[
\alpha=4E^{2};\quad\beta=4(E^{2}-1+EL^{2});\quad\gamma=4EL^{2}+L^{4};\quad\delta=L^{4}.
\]
Then the polynomial has the form $\alpha X^{3}+\beta X^{2}+\gamma X+\delta$.
Suppose $E>0$. Setting 
\[
X=Y-\frac{\beta}{3\alpha};\quad p=\frac{3\alpha\gamma-\beta^{2}}{3\alpha^{2}};\quad q=\frac{2\beta^{3}-9\alpha\beta\gamma+27\alpha^{2}\delta}{27\alpha^{3}},
\]
we have that $\alpha X^{3}+\beta X^{2}+\gamma X+\delta=0$ iff $Y^{3}+pY+q=0$
where the roots of the two polynomials are linked by the formula $X_{i}=Y_{i}-\frac{\beta}{3\alpha}.$
As for the discriminant of the $Y$ polynomial, it is 
\[
\Delta=-(4p^{3}+27q^{2}).
\]
Since only its sign matter on looking for the behavior of the solutions,
we may only compute $\Delta$.

If $\Delta<0$, i.e. $4p^{3}+27q^{2}>0$, then the polynomial has
only one real root given by
\[
Y=\left(-\frac{q}{2}+\sqrt{\frac{q^{2}}{4}+\frac{p^{3}}{27}}\right)^{1/3}+\left(-\frac{q}{2}-\sqrt{\frac{q^{2}}{4}+\frac{p^{3}}{27}}\right)^{1/3}
\]

If $\Delta\geq0$, i.e. $4p^{3}+27q^{2}\geq0$, then there are three
real roots given by
\[
Y_{k}=2\sqrt{-\frac{p}{3}}\cos\left[\frac{1}{3}\arccos\left(\frac{3q}{2p}\sqrt{-\frac{3}{p}}\right)-\frac{2\pi k}{3}\right],\quad k\in\{0,1,2\}.
\]
Note that when $\Delta=0$, this reduces to the two roots 
\[
Y_{0}=2\sqrt{-\frac{p}{3}},\quad Y_{1,2}=-\sqrt{-\frac{p}{3}}.
\]
 Recall $\dot{r}^{2}/2=\psi_{{\rm eff}}(r;L)-E$ ($\geq0$ on the
orbit). $\psi_{{\rm eff}}(r;L)$ has derivative is
\[
\psi'_{{\rm eff}}(r;L)=\frac{L^{2}}{r^{3}}-\frac{r}{(1+r^{2})^{3/2}}
\]
with $\psi'_{{\rm eff}}(r;L)\leq0$ iif $r^{4}/L^{2}\geq(1+r^{2})^{3/2}$.
Left term grows more quickly $(\sim r^{4})$ that the second term
$(\sim r^{3})$ but stars at 0 whereas second term starts at $1/2>0$.
Therefore the two curves cross at a unique point, and this inequality
is satisfied after this point. This shows that $\psi_{{\rm eff}}(r;L)$
is increasing until some point, then decreases. In this maximum is
stricly below $E$, then there are no solution. If the maximum is
exactly $E$, then there is only one solution and the orbit is circular.
If the maximum is strictly above $E$, then they are two solutions
which are $r_{{\rm min}}$and $r_{\max}$. The latter is because $\lim_{r\rightarrow\infty}\psi_{{\rm eff}}(r;L)=0<E$. 

If we are in the case of no solution, then $v_{r}(r)=2(\psi_{{\rm eff}}(r;L)-E)<0$,
which is impossible on an orbit.

If we are in the case with two solutions, then the solution are not
the maximum of $\psi_{{\rm eff}}(r;L)$, meaning that the derivative
evaluated at the solutions are non-zero. This completes the proof
that $1/v_{r}(r)$ is integrable at $r_{\max}$ and $r_{\min}$ as
the integral of $1/\sqrt{r-r_{{\rm m}}}$. Furthermore, since $\psi_{{\rm eff}}(r;L)$
must have two positive distincts solutions $r_{\min},r_{\max}$, then
the polynomial should also have two distincts positive solutions $X_{{\rm max}}=r_{\max}^{2}$
and $X_{\min}=r_{\min}^{2}$ (and the third one being negative).

One should check whether a given couple $(E,L)$ allows for bound
orbits. That that end, we should find if there exists at least one
$r$ that that $\psi_{{\rm eff}}(r;L)\geq E$, i.e. if there are solution
to the polynomial. Equivalently, this reduces to computing the discriminant
$\Delta$ and testing if it is positive.

\subsubsection{Computing the binding energy of a circular orbit given $L$}

We have now access to the NR, orbit-averaged diffusion coefficients
in $(E,L)$-space for the allowed bound orbits: $\bar{D}_{E}$, $\bar{D}_{EE}$,
$\bar{D}_{L}$, $\bar{D}_{LL}$ and $\bar{D}_{EL}$, functions of
$(E,L)$. The allowed region in $(E,L)$ space is composed of the
$E,L\geq0$ such that there exists $r>0$ verifying the inequality
$\psi_{{\rm eff}}(r;L)\geq E$, where we defined the effective potential
\[
\psi_{{\rm eff}}(r;L)=\psi(r)-\frac{L^{2}}{2r^{2}}=\frac{1}{\sqrt{1+r^{2}}}-\frac{L^{2}}{2r^{2}}.
\]
As shown before, for $L>0$, this function has limits $\lim_{0}\psi_{{\rm eff}}=-\infty$
and $\lim_{+\infty}\psi_{{\rm eff}}=0$, is increasing until a global
maximum before decreasing towards $0$. Raising the value of $L$
lowers this maximum value, meaning that there exists a value $L_{{\rm c}}(E)$
such that $\psi_{{\rm eff}}(r;L_{c}(E))=E$. Then, the forbidden angular
momenta (for a given $E$) are the $L>L_{{\rm c}}(E)$. Due to the
discussion in the previous section, this couple $(E,L_{{\rm c}}(E))$
determines a circular orbit. 

There are a priori no analytical formula composed only of basic operations
and radical for $L_{{\rm c}}(E)$. Indeed, we noted that an orbit
with $(E,L)$ was circular if the discriminant $\Delta=18\alpha\beta\gamma\delta-4\beta^{3}\delta+\beta^{2}\gamma^{2}-4\alpha\gamma^{3}-27\alpha^{2}\delta^{2}$,
where $\alpha=4E^{2};\beta=4(E^{2}-1+EL^{2});\gamma=4EL^{2}+L^{4};\delta=L^{4}$,
was zero. This is a degree-6 polynomial equation in the variable $L^{2}$,
which has no such formula for its solutions (Abel, 1826). However,
we may approximate it. For simplicity's sake, look for $E_{c}(L)$
at a given $L$. It is given by $E_{c}(L)=\max_{r>0}\psi_{{\rm eff}}(r;L)=\psi_{{\rm eff}}(r_{*}^{L};L)$.
To approximate this $r_{*}^{L}$, we may look for it using Newton's
method applied to $\psi'_{{\rm eff}}$, since $\psi'_{{\rm eff}}(r_{*}^{L};L)=0$.
Start at $r_{0}^{L}=L^{2/3}$ , where the evaluation yields
\[
\psi'_{{\rm eff}}(L^{2/3};L)=1-\frac{L^{2/3}}{(1+L^{4/3})^{3/2}}\in[1-\sqrt{4/27},1]\simeq[0.615,1]
\]
in order not to be too far away from $\psi'_{{\rm eff}}(r_{*}^{L};L)=0$,
and apply the recursion
\[
r_{n+1}^{L}=r_{n}^{L}-\frac{\psi'_{{\rm eff}}(r_{n}^{L};L)}{\psi''_{{\rm eff}}(r_{n}^{L};L)},
\]
where
\[
\psi'_{{\rm eff}}(r;L)=-\frac{r}{(1+r^{2})^{3/2}}+\frac{L^{2}}{r^{3}};\quad\psi''_{{\rm eff}}(r_{n};L)=-\frac{(1+r^{2})^{3/2}-3r^{2}\sqrt{1+r^{2}}}{(1+r^{2})^{3}}-3\frac{L^{2}}{r^{4}}.
\]
Then $r_{n}^{L}\rightarrow r_{*}^{L}$. We can show that $(r_{n}^{L})$
is increasing since $\psi'_{{\rm eff}}(r_{n}^{L};L)>0$ and $\psi''_{{\rm eff}}(r_{n}^{L};L)<0$
(and convexity of $\psi'_{{\rm eff}}$ where it matters). Therefore
a good stopping condition is to get the lowest $N$ such that $\psi'_{{\rm eff}}(r_{N}^{L}+\epsilon)<0$
for some precision $\epsilon>0$. Then, taking $\tilde{r}_{*}^{L}=(r_{N}^{L}+r_{N}^{L}+\epsilon)/2=r_{N}^{L}+\epsilon/2$
we will have $E_{{\rm c}}(L)\simeq\psi_{{\rm eff}}(\tilde{r}_{*}^{L};L)$,
with precision 
\[
\delta E_{{\rm c}}(L)\simeq|\psi_{{\rm eff}}(r_{*}^{L};L)-\psi_{{\rm eff}}(\tilde{r}_{*}^{L};L)|\simeq\frac{1}{2}|\underbrace{\psi_{{\rm eff}}^{(2)}(r_{*}^{L})}_{<0}|\cdot|r_{*}^{L}-\tilde{r}_{*}^{L}|^{2}\simeq|\psi_{{\rm eff}}^{(2)}(r_{*}^{L})|\frac{\epsilon^{2}}{8}
\]


\subsubsection{Computing the angular momentum of a circular orbit given $E$}

We use the same technique as before. For any $r$, $E$, $L$ possible,
\[
L^{2}\leq2r^{2}(\psi(r)-E)\doteqdot z(r;E)
\]
with equality for the radii $r$ such that $v_{r}(r)=0$, that is,
the bounds of motion. $z(r;E)$ has two solutions, as shown in the
previous analysis, and one global maximum on $r>0$. For a given $E$,
the maximum $L_{{\rm c}}(E)$ that a test star can have is such that
only one $r_{*}^{E}$ satisties $L_{{\rm c}}^{2}(E)\leq z(r_{*}^{E};E)$,
that is, 
\[
L_{{\rm c}}^{2}(E)=z(r_{*}^{E};E)=\max_{r>0}z(r;E).
\]
This is therefore the angular momentum of the circular orbit corresponding
to the binding energy $E$. Given the behavior of $z(r;E)$ (see appendix),
we can use the similar Newton's method to find $r_{*}^{E}$ starting
from $r_{0}^{E}=\sqrt{E^{-2}-1}$, applying the recursion
\[
r_{n+1}^{E}=r_{n}^{E}-\frac{z'(r_{n}^{E};E)}{z''(r_{n}^{E};E)},
\]
and stopping at the lowest $N$ such that $z'(r_{N}^{E}-\epsilon;E)>0$
for some precision $\epsilon>0$. Then, 
\[
L_{{\rm c}}(E)\simeq\sqrt{z(r_{N}^{E}-\frac{\epsilon}{2};E)}
\]
We have that 
\[
z'(r;E)=4r(\psi(r)-E)+2r^{2}\psi'(r)
\]
\[
z''(r;E)=4(\psi(r)-E)+8r\psi'(r)+2r^{2}\psi''(r)
\]


\subsubsection{Study of the radiux acceleration $\ddot{r}$ for non-circular orbits}

A condition to the periodicity of $r(t)$ is that $\ddot{r}(t)$ should
be strictly negative when reaching $r_{\max}$ and strictly positive
when reaching $r_{\min}$, so that for small $\epsilon>0$ (time just
after $t_{\max}$ and $t_{\min}$)
\[
r(t_{\max}+\epsilon)=r(t_{\max})+\dot{r}(t_{\max})\epsilon+\ddot{r}(t_{\max})\frac{\epsilon^{2}}{2}+o(\epsilon^{2})=r_{\max}+\ddot{r}_{\max}\frac{\epsilon^{2}}{2}+o(\epsilon^{2})<r_{\max},
\]
\[
r(t_{\min}+\epsilon)=r(t_{\min})+\dot{r}(t_{\min})\epsilon+\ddot{r}(t_{\min})\frac{\epsilon^{2}}{2}+o(\epsilon^{2})=r_{\min}+\ddot{r}_{\min}\frac{\epsilon^{2}}{2}+o(\epsilon^{2})>r_{\min}.
\]
Newton's law asserts that $a_{r}=\ddot{r}-r\dot{\theta}^{2}=\psi'(r)$,
where $L=r^{2}\dot{\theta}$, hence $\ddot{r}=\psi'(r)+\frac{L^{2}}{r^{3}}=\psi'_{{\rm eff}}(r;L)$.
In the case of a non-circular orbits, $r_{\max}$ and $r_{\min}$
don't correspond to maxima of $\psi{}_{{\rm eff}}(r;L)$, therefore
$\psi'_{{\rm eff}}(r;L)$ doesn't vanish at those points. Furthermore,
we showed that $\psi{}_{{\rm eff}}(r;L)$ was strictly increasing
until its maximum, then was strictly decreasing. This proves that
$\ddot{r}_{\min}>0$ and that $\ddot{r}_{\max}<0$.

\subsection{Diffusion equation and change of variables}

All in all, those coefficients appear in the Fokker-Planck diffusion
equation
\[
\frac{\partial P}{\partial t}(E,L,t)=-\frac{\partial}{\partial E}\left[\bar{D}_{E}P\right]+\frac{1}{2}\frac{\partial^{2}}{\partial E^{2}}\left[\bar{D}_{EE}P\right]-\frac{\partial}{\partial L}\left[\bar{D}_{L}P\right]+\frac{1}{2}\frac{\partial^{2}}{\partial L^{2}}\left[\bar{D}_{LL}P\right]+\frac{1}{2}\frac{\partial^{2}}{\partial E\partial L}\left[\bar{D}_{EL}P\right]
\]

If we want to change coordinates $\boldsymbol{x}\rightarrow\boldsymbol{x'}(\boldsymbol{x})$,
we may use the formulae (C.52) and (C.53) p.25 from Bar-Or \& Alexander
(2016)
\[
D'_{l}=\frac{\partial x'_{l}}{\partial x_{k}}D_{k}+\frac{1}{2}\frac{\partial^{2}x'_{l}}{\partial x_{r}\partial x_{k}}D_{rk},
\]
(error on the sign? should be
\[
D'_{l}=\frac{\partial x'_{l}}{\partial x_{k}}D_{k}-\frac{1}{2}\frac{\partial^{2}x'_{l}}{\partial x_{r}\partial x_{k}}D_{rk}\quad?)
\]
\[
D'_{lm}=\frac{\partial x'_{l}}{\partial x_{r}}\frac{\partial x'_{m}}{\partial x_{k}}D_{rk}.
\]
We might be interested in the change of variable $(E,L)\rightarrow(E,\ell)$
where $\ell(E,L)=L/L_{{\rm c}}(E)$. We have
\begin{center}
\begin{tabular}{|c|c|}
\hline 
 $(E,L)$ &  $(E,\ell)$\tabularnewline
\hline 
\hline 
 $(0,0)$ &  $(0,0)$\tabularnewline
\hline 
 $(E,0)$ &  $(E,0)$\tabularnewline
\hline 
 $(1-\epsilon,\alpha)$ &  $(1-\epsilon,\alpha/L_{{\rm c}}(1-\epsilon))$\tabularnewline
\hline 
 $(E,L_{{\rm c}}(E))$ &  $(E,1)$\tabularnewline
\hline 
 $(0,L>0)$ &  $(0,1)$\tabularnewline
\hline 
 $(0,+\infty)$ &  $(0,1)$\tabularnewline
\hline 
\end{tabular}
\par\end{center}

For $(E,L)=(1-\epsilon,\alpha)$ with $\epsilon,\alpha>0$ small,
letting $\alpha\rightarrow0$ more quickly than $\epsilon\rightarrow0$
corresponds to taking an arbitry limit in the authorized $(E,L)$-space
towards $(E,L)=(1,0)$. Then $L_{{\rm c}}(1-\epsilon)\sim|L_{{\rm c}}(1)|\epsilon$
and $(E,\ell)\rightarrow(1,0)$, meaning that we have
\begin{center}
\begin{tabular}{|c|c|}
\hline 
 $(E,L)$ &  $(E,\ell)$\tabularnewline
\hline 
\hline 
 $(0,0)$ &  $(0,0)$\tabularnewline
\hline 
 $(E,0)$ &  $(E,0)$\tabularnewline
\hline 
 $(1,0)$ &  $(1,0)$\tabularnewline
\hline 
 $(E,L_{{\rm c}}(E))$ &  $(E,1)$\tabularnewline
\hline 
 $(0,L)$ &  $(0,\ell)$\tabularnewline
\hline 
 $(0,+\infty)$ &  $(0,1)$\tabularnewline
\hline 
\end{tabular}
\par\end{center}

We have thus transformed the $(E,L)$-space to a square $(E,\ell)$-space,
were:
\begin{itemize}
\item the side $\ell=0$ is a degenerated bound orbit (straight line through
the center with $r_{\max}=\sqrt{1/E^{2}-1}$. Because of the definition,
its period is $4\int_{0}^{r_{\max}}{\rm d}r/\sqrt{2(\psi_{{\rm eff}}(r)-E}$
($r_{\max}\rightarrow0\rightarrow r_{\max}\rightarrow0\rightarrow r_{\max}$
instead of $r_{\max}\rightarrow r_{\min}\rightarrow r_{\max}$).
\item the side $\ell=1$ is the limit of circular orbits.
\item the side $E=0$ is the limit of unbounded orbits.
\item the side $E=1$ is the limit of forbidden parameters.
\end{itemize}
Then we obtain the transformed Fokker-Planck equation
\[
\frac{\partial P}{\partial t}(E,\ell,t)=-\frac{\partial}{\partial E}\left[\bar{D}_{E}P\right]+\frac{1}{2}\frac{\partial^{2}}{\partial E^{2}}\left[\bar{D}_{EE}P\right]-\frac{\partial}{\partial\ell}\left[\bar{D}_{\ell}P\right]+\frac{1}{2}\frac{\partial^{2}}{\partial\ell^{2}}\left[\bar{D}_{\ell\ell}P\right]+\frac{1}{2}\frac{\partial^{2}}{\partial E\partial\ell}\left[\bar{D}_{E\ell}P\right]
\]


\subsection{Useful Julia packages}
\begin{itemize}
\item HCuba, with its function cuhre((x,f)->f{[}1{]} = integrand, 3, 1),
to integrable multidimensional integrals (here 3D) over the unit hypercube.
May be useful to compute the local diffusion coefficients since all
integration bounds are finite. The cuhre() function is deterministic,
fast and globally adaptive.
\end{itemize}

\section{Brouillon}

\subsection{Local diffusion coefficients}

Rep�re fix� quelconque $e_{1}=(Oz)$, $e_{2},e_{3}$. Vecteur $\boldsymbol{r}=(r,\theta,\phi)$,
fix�. 
\[
r_{1}=r\cos\theta
\]
\[
r_{2}=r\sin\theta\cos\phi
\]
\[
r_{3}=r\sin\theta\sin\phi
\]
Vecteur vitesse de la particule test $\boldsymbol{v}$
\[
v_{1}=v\cos\theta_{v}
\]
\[
v_{2}=v\sin\theta_{v}\cos\phi_{v}
\]
\[
v_{3}=v\sin\theta_{v}\sin\phi_{v}
\]
en cylindrique avec $\theta_{v}$ l'angle entre $\hat{e_{1}}$ et
$\boldsymbol{v}$, et $\phi_{v}$ l'angle entre $\hat{e_{2}}$ et
le projet� de $\boldsymbol{v}$ sur le plan equatorial. On a $\boldsymbol{v}=v_{r}\hat{r}+\boldsymbol{v_{t}}=\boldsymbol{v_{r}}+\boldsymbol{v_{t}}$.
\[
v_{r}=\boldsymbol{v}\cdot\hat{r}=v_{1}\hat{r}_{1}+v_{2}\hat{r}_{2}+v_{3}\hat{r}_{3}=v_{1}\cos\theta+v_{2}\sin\theta\cos\phi+v_{3}\sin\theta\sin\phi
\]

\[
v_{t}^{2}=v^{2}-(v_{1}\cos\theta+v_{2}\sin\theta\cos\phi+v_{3}\sin\theta\sin\phi)
\]
\[
v_{t}^{2}=v^{2}-v_{r}^{2}=(\underbrace{v_{1}^{2}+v_{2}^{2}+v_{3}^{2}}_{=v^{2}})-(\underbrace{v_{1}\cos\theta+v_{2}\sin\theta\cos\phi+v_{3}\sin\theta\sin\phi}_{=v_{r}})^{2}
\]

. Let $h(\boldsymbol{r},\boldsymbol{v})=h(r,v_{r},v_{t})$. Note that
$\frac{\partial r}{\partial v_{i}}=0$. Thus
\[
\frac{\partial h}{\partial v_{1}}=\frac{\partial v_{r}}{\partial v_{1}}\frac{\partial h}{\partial v_{r}}+\frac{\partial v_{t}}{\partial v_{1}}\frac{\partial h}{\partial v_{t}}
\]
We have, (at fixed $\boldsymbol{r}$(Fixed $\boldsymbol{r}$ mean
fixed $r,\theta,\phi$)).
\[
\frac{\partial v_{r}}{\partial v_{1}}=\cos\theta
\]
and 
\[
\frac{\partial(v_{t}^{2})}{\partial v_{1}}=2v_{t}\frac{\partial v_{t}}{\partial v_{1}}=\frac{\partial(v^{2})}{\partial v_{1}}-\frac{\partial(v_{r}^{2})}{\partial v_{1}}=2v_{1}-2v_{r}\frac{\partial v_{r}}{\partial v_{1}}=2v_{1}-2v_{r}\cos\theta
\]
hence
\[
\frac{\partial v_{t}}{\partial v_{1}}=\frac{v_{1}-v_{r}\cos\theta}{v_{t}}
\]
In the coordinate system where $\hat{e_{1}}=\hat{v}$ and $\hat{e_{2}}$
is the projection of $\hat{r}$ on the equatorial plane ($\phi=0$),
we have that $\Delta v_{1}=\Delta v_{||}$ and $\Delta v_{2}^{2}+\Delta v_{3}^{2}=\Delta v_{\perp}^{2}$,
meaning that 
\[
\langle\Delta v_{||}\rangle=\langle\Delta v_{1}\rangle
\]
\[
\langle(\Delta v_{||})^{2}\rangle=\langle(\Delta v_{1})^{2}\rangle
\]
\[
\langle(\Delta v_{\perp})^{2}\rangle=\langle(\Delta v_{2})^{2}\rangle+\langle(\Delta v_{3})^{2}\rangle
\]
In that system, we have that $\theta_{v}=0$, $v_{1}=v$ and $\phi$We
also have that $\cos\theta=v_{r}/v$ because $\theta$ is the angle
between $\hat{r}$ and $(Oz)=\hat{e_{1}}=\hat{v}$, and $v_{r}$ is
the orthogonal projection of $\boldsymbol{v}$ on $\hat{r}$. For
the same reason, $\sin\theta=v_{t}/v$. Therefore, in that coordinate
system:
\[
\frac{\partial v_{r}}{\partial v_{1}}=\frac{v_{r}}{v};\quad\frac{\partial v_{t}}{\partial v_{1}}=\frac{v-v_{r}^{2}/v}{v_{t}}=\frac{v^{2}-v_{r}^{2}}{v_{t}v}=\frac{v_{t}^{2}}{v_{t}v}=\frac{v_{t}}{v}
\]
and we have
\[
\frac{\partial h}{\partial v_{1}}=\frac{v_{r}}{v}\frac{\partial h}{\partial v_{r}}+\frac{v_{t}}{v}\frac{\partial h}{\partial v_{t}}
\]

As for second order:
\[
\frac{\partial^{2}g}{\partial v_{1}^{2}}=\frac{\partial}{\partial v_{1}}\left(\frac{\partial g}{\partial v_{1}}\right)=\frac{\partial}{\partial v_{1}}\left(\frac{\partial v_{r}}{\partial v_{1}}\frac{\partial g}{\partial v_{r}}+\frac{\partial v_{t}}{\partial v_{1}}\frac{\partial g}{\partial v_{t}}\right)
\]
with $\frac{\partial v_{r}}{\partial v_{1}}=\cos\theta$ and $\frac{\partial v_{t}}{\partial v_{1}}=\frac{v_{1}-v_{r}\cos\theta}{v_{t}}$.
Then
\begin{align*}
\frac{\partial^{2}g}{\partial v_{1}^{2}} & =\frac{\partial}{\partial v_{1}}\left(\frac{\partial v_{r}}{\partial v_{1}}\right)\frac{\partial g}{\partial v_{r}}+\left(\frac{\partial v_{r}}{\partial v_{1}}\right)^{2}\frac{\partial^{2}g}{\partial v_{r}^{2}}+\left(\frac{\partial v_{r}}{\partial v_{1}}\frac{\partial v_{t}}{\partial v_{1}}\right)\frac{\partial^{2}g}{\partial v_{t}\partial v_{r}}\\
+ & \frac{\partial}{\partial v_{1}}\left(\frac{\partial v_{t}}{\partial v_{1}}\right)\frac{\partial g}{\partial v_{t}}+\left(\frac{\partial v_{t}}{\partial v_{1}}\right)^{2}\frac{\partial^{2}g}{\partial v_{t}^{2}}+\left(\frac{\partial v_{t}}{\partial v_{1}}\frac{\partial v_{r}}{\partial v_{1}}\right)\frac{\partial^{2}g}{\partial v_{t}\partial v_{r}}
\end{align*}
where 
\[
\frac{\partial}{\partial v_{1}}\left(\frac{\partial v_{r}}{\partial v_{1}}\right)\frac{\partial g}{\partial v_{r}}=\frac{\partial}{\partial v_{1}}\left(\cos\theta\right)\frac{\partial g}{\partial v_{r}}=0
\]
\[
\left(\frac{\partial v_{r}}{\partial v_{1}}\right)^{2}\frac{\partial^{2}g}{\partial v_{r}^{2}}=\cos^{2}\theta\frac{\partial^{2}g}{\partial v_{r}^{2}}
\]
\[
\left(\frac{\partial v_{r}}{\partial v_{1}}\frac{\partial v_{t}}{\partial v_{1}}\right)\frac{\partial^{2}g}{\partial v_{t}\partial v_{r}}=\left(\cos\theta\frac{v_{1}-v_{r}\cos\theta}{v_{t}}\right)\frac{\partial^{2}g}{\partial v_{t}\partial v_{r}}
\]
\begin{align*}
\frac{\partial}{\partial v_{1}}\left(\frac{\partial v_{t}}{\partial v_{1}}\right)\frac{\partial g}{\partial v_{t}} & =\frac{\partial}{\partial v_{1}}\left(\cos\theta\frac{v_{1}-v_{r}\cos\theta}{v_{t}}\right)\frac{\partial g}{\partial v_{t}}=\cos\theta\left(\frac{(1-\frac{\partial v_{r}}{\partial v_{1}}\cos\theta)v_{t}-(v_{1}-v_{r}\cos\theta)\frac{\partial v_{t}}{\partial v_{1}}}{v_{t}^{2}}\right)\frac{\partial g}{\partial v_{t}}\\
 & =\cos\theta\left(\frac{(1-\cos^{2}\theta)v_{t}-(v_{1}-v_{r}\cos\theta)\frac{v_{1}-v_{r}\cos\theta}{v_{t}}}{v_{t}^{2}}\right)\frac{\partial g}{\partial v_{t}}\\
 & =\cos\theta\left(\frac{\sin^{2}\theta v_{t}-(v_{1}-v_{r}\cos\theta)\frac{v_{1}-v_{r}\cos\theta}{v_{t}}}{v_{t}^{2}}\right)\frac{\partial g}{\partial v_{t}}
\end{align*}

\[
\left(\frac{\partial v_{t}}{\partial v_{1}}\right)^{2}\frac{\partial^{2}g}{\partial v_{t}^{2}}=\left(\frac{v_{1}-v_{r}\cos\theta}{v_{t}}\right)^{2}\frac{\partial^{2}g}{\partial v_{t}^{2}}
\]
Which in the special coordinate system gives
\[
\frac{\partial}{\partial v_{1}}\left(\frac{\partial v_{r}}{\partial v_{1}}\right)\frac{\partial g}{\partial v_{r}}=0
\]
\[
\left(\frac{\partial v_{r}}{\partial v_{1}}\right)^{2}\frac{\partial^{2}g}{\partial v_{r}^{2}}=\frac{v_{r}^{2}}{v^{2}}\frac{\partial^{2}g}{\partial v_{r}^{2}}
\]
\[
\left(\frac{\partial v_{r}}{\partial v_{1}}\frac{\partial v_{t}}{\partial v_{1}}\right)\frac{\partial^{2}g}{\partial v_{t}\partial v_{r}}=\left(\frac{v_{r}}{v}\frac{v-v_{r}^{2}/v}{v_{t}}\right)\frac{\partial^{2}g}{\partial v_{t}\partial v_{r}}=\left(\frac{v_{r}v_{t}}{v^{2}}\right)\frac{\partial^{2}g}{\partial v_{t}\partial v_{r}}
\]
\begin{align*}
\frac{\partial}{\partial v_{1}}\left(\frac{\partial v_{t}}{\partial v_{1}}\right)\frac{\partial g}{\partial v_{t}} & =\frac{v_{r}}{v}\left(\frac{v_{t}^{3}/v^{2}-(v-v_{r}^{2}/v)\frac{v-v_{r}^{2}/v}{v_{t}}}{v_{t}^{2}}\right)\frac{\partial g}{\partial v_{t}}\\
 & \frac{\partial}{\partial v_{1}}\left(\frac{\partial v_{t}}{\partial v_{1}}\right)\frac{\partial g}{\partial v_{t}}==\frac{v_{r}}{v}\left(\frac{v_{t}^{3}/v^{2}-\frac{v_{t}^{3}}{v^{2}}}{v_{t}^{2}}\right)\frac{\partial g}{\partial v_{t}}=0
\end{align*}

\[
\left(\frac{\partial v_{t}}{\partial v_{1}}\right)^{2}\frac{\partial^{2}g}{\partial v_{t}^{2}}=\left(\frac{v-v_{r}^{2}/v}{v_{t}}\right)^{2}\frac{\partial^{2}g}{\partial v_{t}^{2}}=\left(\frac{v_{t}}{v}\right)^{2}\frac{\partial^{2}g}{\partial v_{t}^{2}}
\]
Hence:
\begin{align*}
\frac{\partial^{2}g}{\partial v_{1}^{2}} & =\frac{v_{r}^{2}}{v^{2}}\frac{\partial^{2}g}{\partial v_{r}^{2}}+\frac{2v_{r}v_{t}}{v^{2}}\frac{\partial^{2}g}{\partial v_{t}\partial v_{r}}+\left(\frac{v_{t}}{v}\right)^{2}\frac{\partial^{2}g}{\partial v_{t}^{2}}
\end{align*}

~

~

~

~

~

~

\[
\frac{\partial^{2}g}{\partial v_{2}^{2}}=\frac{\partial}{\partial v_{2}}\left(\frac{\partial g}{\partial v_{2}}\right)=\frac{\partial}{\partial v_{2}}\left(\frac{\partial v_{r}}{\partial v_{2}}\frac{\partial g}{\partial v_{r}}+\frac{\partial v_{t}}{\partial v_{2}}\frac{\partial g}{\partial v_{t}}\right)
\]
with 
\[
\frac{\partial v_{r}}{\partial v_{2}}=\sin\theta\cos\phi
\]
\[
\frac{\partial(v_{t}^{2})}{\partial v_{2}}=2v_{t}\frac{\partial v_{t}}{\partial v_{2}}=\frac{\partial(v^{2})}{\partial v_{2}}-\frac{\partial(v_{r}^{2})}{\partial v_{2}}=2v_{2}-2v_{r}\frac{\partial v_{r}}{\partial v_{2}}=2v_{2}-2v_{r}\sin\theta\cos\phi
\]
thus
\[
\frac{\partial v_{t}}{\partial v_{2}}=\frac{v_{2}-v_{r}\sin\theta\cos\phi}{v_{t}}
\]
thus
\begin{align*}
\frac{\partial^{2}g}{\partial v_{2}^{2}} & =\left(\frac{\partial}{\partial v_{2}}\frac{\partial v_{r}}{\partial v_{2}}\right)\frac{\partial g}{\partial v_{r}}+\left(\frac{\partial v_{r}}{\partial v_{2}}\right)^{2}\frac{\partial^{2}g}{\partial v_{r}^{2}}+\left(\frac{\partial v_{r}}{\partial v_{2}}\frac{\partial v_{t}}{\partial v_{2}}\right)\frac{\partial^{2}g}{\partial v_{t}\partial v_{r}}\\
 & +\left(\frac{\partial}{\partial v_{2}}\frac{\partial v_{t}}{\partial v_{2}}\right)\frac{\partial g}{\partial v_{t}}+\left(\frac{\partial v_{t}}{\partial v_{2}}\frac{\partial v_{r}}{\partial v_{2}}\right)\frac{\partial^{2}g}{\partial r\partial v_{t}}+\left(\frac{\partial v_{t}}{\partial v_{2}}\right)^{2}\frac{\partial^{2}g}{\partial v_{t}^{2}}
\end{align*}
we have
\[
\frac{\partial}{\partial v_{2}}\frac{\partial v_{r}}{\partial v_{2}}=\frac{\partial}{\partial v_{2}}(\sin\theta\cos\phi)=0
\]
\[
\left(\frac{\partial v_{r}}{\partial v_{2}}\right)^{2}=\sin^{2}\theta\cos^{2}\phi
\]
\[
\frac{\partial v_{r}}{\partial v_{2}}\frac{\partial v_{t}}{\partial v_{2}}=\sin\theta\cos\phi\frac{v_{2}-v_{r}\sin\theta\cos\phi}{v_{t}}
\]
\begin{align*}
\frac{\partial}{\partial v_{2}}\frac{\partial v_{t}}{\partial v_{2}} & =\frac{\partial}{\partial v_{2}}(\frac{v_{2}-v_{r}\sin\theta\cos\phi}{v_{t}})=\frac{(1-\frac{\partial v_{r}}{\partial v_{2}}\sin\theta\cos\phi)v_{t}-(v_{2}-v_{r}\sin\theta\cos\phi)\frac{\partial v_{t}}{\partial v_{2}}}{v_{t}^{2}}\\
 & =\frac{(1-\sin^{2}\theta\cos^{2}\phi)v_{t}-(v_{2}-v_{r}\sin\theta\cos\phi)\frac{v_{2}-v_{r}\sin\theta\cos\phi}{v_{t}}}{v_{t}^{2}}
\end{align*}
\[
\left(\frac{\partial v_{t}}{\partial v_{2}}\right)^{2}=(\frac{v_{2}-v_{r}\sin\theta\cos\phi}{v_{t}})^{2}
\]
In the special coordinate system:
\[
\frac{\partial}{\partial v_{2}}\frac{\partial v_{r}}{\partial v_{2}}=0
\]
\[
\left(\frac{\partial v_{r}}{\partial v_{2}}\right)^{2}=\left(\frac{v_{t}}{v}\right)^{2}
\]
\[
\frac{\partial v_{r}}{\partial v_{2}}\frac{\partial v_{t}}{\partial v_{2}}=\frac{-v_{r}(v_{t}^{2}/v^{2})}{v_{t}}=-\frac{v_{r}v_{t}}{v^{2}}
\]
\begin{align*}
\frac{\partial}{\partial v_{2}}\frac{\partial v_{t}}{\partial v_{2}} & =\frac{(1-\frac{v_{t}^{2}}{v^{2}})v_{t}-(-v_{r}v_{t}/v)\frac{-v_{r}v_{t}/v}{v_{t}}}{v_{t}^{2}}=\frac{1}{v_{t}}(1-\frac{v_{t}^{2}}{v^{2}})-\frac{v_{r}^{2}}{v^{2}v_{t}}==\frac{v_{r}^{2}}{v^{2}v_{t}}-\frac{v_{r}^{2}}{v^{2}v_{t}}=0
\end{align*}
\[
\left(\frac{\partial v_{t}}{\partial v_{2}}\right)^{2}=\left(\frac{v_{r}}{v}\right)^{2}
\]
\begin{align*}
\frac{\partial^{2}g}{\partial v_{2}^{2}} & =\left(\frac{v_{t}}{v}\right)^{2}\frac{\partial^{2}g}{\partial v_{r}^{2}}-\frac{2v_{r}v_{t}}{v^{2}}\frac{\partial^{2}g}{\partial v_{t}\partial v_{r}}+\left(\frac{v_{r}}{v}\right)^{2}\frac{\partial^{2}g}{\partial v_{t}^{2}}
\end{align*}
And for the third:
\[
\frac{\partial^{2}g}{\partial v_{3}^{2}}=\frac{\partial}{\partial v_{3}}\left(\frac{\partial g}{\partial v_{3}}\right)=\frac{\partial}{\partial v_{3}}\left(\frac{\partial v_{r}}{\partial v_{3}}\frac{\partial g}{\partial v_{r}}+\frac{\partial v_{t}}{\partial v_{3}}\frac{\partial g}{\partial v_{t}}\right)
\]
with
\[
\frac{\partial v_{r}}{\partial v_{3}}=\sin\theta\sin\phi
\]
\[
\frac{\partial(v_{t}^{2})}{\partial v_{3}}=2v_{t}\frac{\partial v_{t}}{\partial v_{3}}=\frac{\partial(v^{2})}{\partial v_{3}}-\frac{\partial(v_{r}^{2})}{\partial v_{3}}=2v_{3}-2v_{r}\frac{\partial v_{r}}{\partial v_{3}}=2v_{3}-2v_{r}\sin\theta\sin\phi
\]
thus
\[
\frac{\partial v_{t}}{\partial v_{3}}=\frac{v_{3}-v_{r}\sin\theta\sin\phi}{v_{t}}
\]
thus
\begin{align*}
\frac{\partial^{2}g}{\partial v_{3}^{2}} & =\left(\frac{\partial}{\partial v_{3}}\frac{\partial v_{r}}{\partial v_{3}}\right)\frac{\partial g}{\partial v_{r}}+\left(\frac{\partial v_{r}}{\partial v_{3}}\right)^{2}\frac{\partial^{2}g}{\partial v_{r}^{2}}+\left(\frac{\partial v_{r}}{\partial v_{3}}\frac{\partial v_{t}}{\partial v_{3}}\right)\frac{\partial^{2}g}{\partial v_{t}\partial v_{r}}\\
 & +\left(\frac{\partial}{\partial v_{3}}\frac{\partial v_{t}}{\partial v_{3}}\right)\frac{\partial g}{\partial v_{t}}+\left(\frac{\partial v_{t}}{\partial v_{3}}\frac{\partial v_{r}}{\partial v_{3}}\right)\frac{\partial^{2}g}{\partial r\partial v_{t}}+\left(\frac{\partial v_{t}}{\partial v_{3}}\right)^{2}\frac{\partial^{2}g}{\partial v_{t}^{2}}
\end{align*}
with
\[
\frac{\partial}{\partial v_{3}}\frac{\partial v_{r}}{\partial v_{3}}=\frac{\partial}{\partial v_{3}}(\sin\theta\sin\phi)=0
\]
\[
\left(\frac{\partial v_{r}}{\partial v_{3}}\right)^{2}=\sin^{2}\theta\sin^{2}\phi
\]
\[
\frac{\partial v_{r}}{\partial v_{3}}\frac{\partial v_{t}}{\partial v_{3}}=\sin\theta\sin\phi\frac{v_{3}-v_{r}\sin\theta\sin\phi}{v_{t}}
\]
\begin{align*}
\frac{\partial}{\partial v_{3}}\frac{\partial v_{t}}{\partial v_{3}} & =\frac{\partial}{\partial v_{3}}\frac{v_{3}-v_{r}\sin\theta\sin\phi}{v_{t}}=\frac{(1-\frac{\partial v_{r}}{\partial v_{3}}\sin\theta\sin\phi)v_{t}-(v_{3}-v_{r}\sin\theta\sin\phi)\frac{\partial v_{t}}{\partial v_{3}}}{v_{t}^{2}}\\
 & =\frac{(1-\sin^{2}\theta\sin^{2}\phi)v_{t}-(v_{3}-v_{r}\sin\theta\sin\phi)\frac{v_{3}-v_{r}\sin\theta\sin\phi}{v_{t}}}{v_{t}^{2}}
\end{align*}
\[
\left(\frac{\partial v_{t}}{\partial v_{3}}\right)^{2}=(\frac{v_{3}-v_{r}\sin\theta\sin\phi}{v_{t}})^{2}
\]
In the special coordinate system
\[
\left(\frac{\partial v_{r}}{\partial v_{3}}\right)^{2}=0
\]
\[
\frac{\partial v_{r}}{\partial v_{3}}\frac{\partial v_{t}}{\partial v_{3}}=0
\]
\begin{align*}
\frac{\partial}{\partial v_{3}}\frac{\partial v_{t}}{\partial v_{3}} & =\frac{1}{v_{t}}
\end{align*}
\[
\left(\frac{\partial v_{t}}{\partial v_{3}}\right)^{2}=0
\]
therefore
\begin{align*}
\frac{\partial^{2}g}{\partial v_{3}^{2}} & =\frac{1}{v_{t}}\frac{\partial g}{\partial v_{t}}
\end{align*}

~

$h(v_{r},v_{t})=h(v)$, $v^{2}=v_{r}^{2}+v_{t}^{2}\rightarrow v\frac{\partial v}{\partial v_{i}}=2v_{i}\rightarrow\frac{\partial v}{\partial v_{i}}=\frac{v_{i}}{v}$
\[
\frac{v_{r}}{v}\frac{\partial h}{\partial v_{r}}+\frac{v_{t}}{v}\frac{\partial h}{\partial v_{t}}=\frac{v_{r}}{v}\frac{\partial v}{\partial v_{r}}h'+\frac{v_{t}}{v}\frac{\partial v}{\partial v_{t}}h'=\frac{v_{r}^{2}}{v^{2}}h'+\frac{v_{t}^{2}}{v^{2}}h'=h'
\]
\begin{align*}
\frac{\partial^{2}g}{\partial v_{r}^{2}} & =\frac{\partial}{\partial v_{r}}\frac{\partial g}{\partial v_{r}}=\frac{\partial}{\partial v_{r}}(\frac{\partial v}{\partial v_{r}}g'(v))=\frac{\partial}{\partial v_{r}}(\frac{v_{r}}{v}g'(v))=\frac{v-v_{r}^{2}/v}{v^{2}}g'(v)+\frac{v_{r}}{v}\frac{\partial v}{\partial v_{r}}g''(v)\\
 & =\frac{v_{t}^{2}}{v^{3}}g'(v)+\frac{v_{r}^{2}}{v^{2}}g''(v)
\end{align*}
\begin{align*}
\frac{\partial^{2}g}{\partial v_{t}\partial v_{r}} & =\frac{\partial}{\partial v_{t}}\frac{\partial g}{\partial v_{r}}=\frac{\partial}{\partial v_{t}}(\frac{\partial v}{\partial v_{r}}g')=\frac{\partial}{\partial v_{t}}(\frac{v_{r}}{v}g')=\frac{-v_{r}v_{t}/v}{v^{2}}g'+\frac{v_{r}}{v}\frac{\partial v}{\partial v_{t}}g''\\
 & =\frac{-v_{r}v_{t}}{v^{3}}g'+\frac{v_{r}v_{t}}{v^{2}}g''
\end{align*}
\begin{align*}
\frac{\partial^{2}g}{\partial v_{t}^{2}} & =\frac{\partial}{\partial v_{t}}\frac{\partial g}{\partial v_{t}}=\frac{\partial}{\partial v_{t}}(\frac{\partial v}{\partial v_{t}}g')=\frac{\partial}{\partial v_{t}}(\frac{v_{t}}{v}g')=\frac{v-v_{t}^{2}/v}{v^{2}}g'+\frac{v_{t}}{v}\frac{\partial v}{\partial v_{t}}g'\\
 & =\frac{v_{r}^{2}}{v^{3}}g'+\frac{v_{t}^{2}}{v^{2}}g''
\end{align*}
\[
\frac{\partial g}{\partial v_{t}}=\frac{\partial v}{\partial v_{t}}g'=\frac{v_{t}}{v}g'
\]
hence
\begin{align*}
\frac{v_{r}^{2}}{v^{2}}\frac{\partial^{2}g}{\partial v_{r}^{2}}+\frac{2v_{r}v_{t}}{v^{2}}\frac{\partial^{2}g}{\partial v_{t}\partial v_{r}}+\left(\frac{v_{t}}{v}\right)^{2}\frac{\partial^{2}g}{\partial v_{t}^{2}} & =\frac{v_{r}^{2}}{v^{2}}\left(\frac{v_{t}^{2}}{v^{3}}g'+\frac{v_{r}^{2}}{v^{2}}g''\right)+\frac{2v_{r}v_{t}}{v^{2}}\left(\frac{-v_{r}v_{t}}{v^{3}}g'+\frac{v_{r}v_{t}}{v^{2}}g''\right)+\left(\frac{v_{t}}{v}\right)^{2}\left(\frac{v_{r}^{2}}{v^{3}}g'+\frac{v_{t}^{2}}{v^{2}}g''\right)\\
 & =\left(\frac{v_{r}^{2}v_{t}^{2}}{v^{5}}-\frac{2v_{r}^{2}v_{t}^{2}}{v^{5}}+\frac{v_{r}^{2}v_{t}^{2}}{v^{5}}\right)g'+\left(\frac{v_{r}^{4}}{v^{4}}+\frac{2v_{r}^{2}v_{t}^{2}}{v^{4}}+\frac{v_{t}^{4}}{v^{4}}\right)g''\\
 & =\left(\frac{v_{r}^{2}}{v^{2}}+\frac{v_{t}^{2}}{v^{2}}\right)^{2}g''=g''
\end{align*}
and
\begin{align*}
\left(\frac{v_{t}}{v}\right)^{2}\frac{\partial^{2}g}{\partial v_{r}^{2}}-\frac{2v_{r}v_{t}}{v^{2}}\frac{\partial^{2}g}{\partial v_{t}\partial v_{r}}+\left(\frac{v_{r}}{v}\right)^{2}\frac{\partial^{2}g}{\partial v_{t}^{2}}+\frac{1}{v_{t}}\frac{\partial g}{\partial v_{t}} & =\left(\frac{v_{t}}{v}\right)^{2}\left(\frac{v_{t}^{2}}{v^{3}}g'+\frac{v_{r}^{2}}{v^{2}}g''\right)-\frac{2v_{r}v_{t}}{v^{2}}\left(\frac{-v_{r}v_{t}}{v^{3}}g'+\frac{v_{r}v_{t}}{v^{2}}g''\right)\\
 & +\left(\frac{v_{r}}{v}\right)^{2}\left(\frac{v_{r}^{2}}{v^{3}}g'+\frac{v_{t}^{2}}{v^{2}}g''\right)+\frac{1}{v_{t}}\frac{v_{t}}{v}g'\\
 & =\left(\frac{v_{t}^{4}}{v^{5}}+\frac{2v_{r}^{2}v_{t}^{2}}{v^{5}}+\frac{v_{r}^{4}}{v^{5}}+\frac{1}{v}\right)g'+\left(\frac{v_{r}^{2}v_{t}^{2}}{v^{4}}-\frac{2v_{r}^{2}v_{t}^{2}}{v^{4}}+\frac{v_{r}^{2}v_{t}^{2}}{v^{4}}\right)g''\\
 & =\frac{1}{v}\left(\left(\frac{v_{t}^{2}}{v^{2}}+\frac{v_{r}^{2}}{v^{2}}\right)^{2}+1\right)g'=\frac{2}{v}g'
\end{align*}


\subsection{Derivatives in the special referential frame}

\[
\frac{\partial h}{\partial v_{1}}=\frac{v_{r}}{v}\frac{\partial h}{\partial v_{r}}+\frac{v_{t}}{v}\frac{\partial h}{\partial v_{t}}
\]

\begin{align*}
\frac{\partial^{2}g}{\partial v_{1}^{2}} & =\frac{v_{r}^{2}}{v^{2}}\frac{\partial^{2}g}{\partial v_{r}^{2}}+\frac{2v_{r}v_{t}}{v^{2}}\frac{\partial^{2}g}{\partial v_{t}\partial v_{r}}+\left(\frac{v_{t}}{v}\right)^{2}\frac{\partial^{2}g}{\partial v_{t}^{2}}
\end{align*}

\begin{align*}
\frac{\partial^{2}g}{\partial v_{2}^{2}} & =\left(\frac{v_{t}}{v}\right)^{2}\frac{\partial^{2}g}{\partial v_{r}^{2}}-\frac{2v_{r}v_{t}}{v^{2}}\frac{\partial^{2}g}{\partial v_{t}\partial v_{r}}+\left(\frac{v_{r}}{v}\right)^{2}\frac{\partial^{2}g}{\partial v_{t}^{2}}
\end{align*}
\begin{align*}
\frac{\partial^{2}g}{\partial v_{3}^{2}} & =\frac{1}{v_{t}}\frac{\partial g}{\partial v_{t}}
\end{align*}

with
\[
\langle\Delta v_{||}\rangle=\langle\Delta v_{1}\rangle
\]
\[
\langle(\Delta v_{||})^{2}\rangle=\langle(\Delta v_{1})^{2}\rangle
\]
\[
\langle(\Delta v_{\perp})^{2}\rangle=\langle(\Delta v_{2})^{2}\rangle+\langle(\Delta v_{3})^{2}\rangle
\]
and
\[
\langle\Delta v_{i}\rangle(r,\boldsymbol{v})=4\pi G^{2}m_{a}(m+m_{a})\ln\Lambda\frac{\partial h}{\partial v_{i}}(r,\boldsymbol{v})
\]
\[
\langle\Delta v_{i}\Delta v_{j}\rangle(r,\boldsymbol{v})=4\pi G^{2}m_{a}^{2}\ln\Lambda\frac{\partial g}{\partial v_{i}\partial v_{j}}(r,\boldsymbol{v})
\]
Therefore
\[
\langle\Delta v_{||}\rangle=4\pi G^{2}m_{a}(m+m_{a})\ln\Lambda\frac{\partial h}{\partial v_{1}}=4\pi G^{2}m_{a}(m+m_{a})\ln\Lambda(\frac{v_{r}}{v}\frac{\partial h}{\partial v_{r}}+\frac{v_{t}}{v}\frac{\partial h}{\partial v_{t}})
\]
\[
\langle(\Delta v_{||})^{2}\rangle=4\pi G^{2}m_{a}^{2}\ln\Lambda\frac{\partial^{2}g}{\partial v_{1}^{2}}=4\pi G^{2}m_{a}^{2}\ln\Lambda\left(\frac{v_{r}^{2}}{v^{2}}\frac{\partial^{2}g}{\partial v_{r}^{2}}+\frac{2v_{r}v_{t}}{v^{2}}\frac{\partial^{2}g}{\partial v_{t}\partial v_{r}}+\left(\frac{v_{t}}{v}\right)^{2}\frac{\partial^{2}g}{\partial v_{t}^{2}}\right)
\]
\begin{align*}
\langle(\Delta v_{\perp})^{2}\rangle & =4\pi G^{2}m_{a}^{2}\ln\Lambda\left(\frac{\partial^{2}g}{\partial v_{2}^{2}}+\frac{\partial^{2}g}{\partial v_{3}^{2}}\right)\\
 & =4\pi G^{2}m_{a}^{2}\ln\Lambda\left(\left(\frac{v_{t}}{v}\right)^{2}\frac{\partial^{2}g}{\partial v_{r}^{2}}-\frac{2v_{r}v_{t}}{v^{2}}\frac{\partial^{2}g}{\partial v_{t}\partial v_{r}}+\left(\frac{v_{r}}{v}\right)^{2}\frac{\partial^{2}g}{\partial v_{t}^{2}}+\frac{1}{v_{t}}\frac{\partial g}{\partial v_{t}}\right)
\end{align*}

Overall, those can be expressed as functions of $h(v_{r},v_{t})$,
$g(v_{r},v_{t})$, $\frac{\partial h}{\partial v_{r}}$, $\frac{\partial h}{\partial v_{t}}$,
$\frac{\partial g}{\partial v_{t}}$, $\frac{\partial^{2}g}{\partial v_{r}^{2}}$,
$\frac{\partial^{2}g}{\partial v_{t}^{2}}$ and$\frac{\partial^{2}g}{\partial v_{r}\partial v_{t}}$.
Consider 
\[
h(r,v_{r},v_{t})=\int{\rm d}^{3}\boldsymbol{v_{a}}\frac{f_{a}(r,\boldsymbol{v_{a}})}{|\boldsymbol{v}-\boldsymbol{v_{a}}|}=\int{\rm d}^{3}\boldsymbol{v'}\frac{f_{a}(r,\boldsymbol{v}-\boldsymbol{v'})}{v'}=\int_{0}^{\infty}{\rm d}v'v'\int_{0}^{\pi}{\rm d}\theta\sin\theta\int_{0}^{2\pi}{\rm d}\phi f_{a}(r,\boldsymbol{v}-\boldsymbol{v'})
\]
\[
g(r,v_{r},v_{t})=\int{\rm d}^{3}\boldsymbol{v_{a}}f_{a}(r,\boldsymbol{v_{a}})|\boldsymbol{v}-\boldsymbol{v_{a}}|=\int{\rm d}^{3}\boldsymbol{v'}f_{a}(r,\boldsymbol{v}-\boldsymbol{v'})v'=\int_{0}^{\infty}{\rm d}v'v'^{3}\int_{0}^{\pi}{\rm d}\theta\sin\theta\int_{0}^{2\pi}{\rm d}\phi f_{a}(r,\boldsymbol{v}-\boldsymbol{v'})
\]
with
\[
\frac{\partial h}{\partial v_{r}}(r,v_{r},v_{t})=\int_{0}^{\infty}{\rm d}v'v'\int_{0}^{\pi}{\rm d}\theta\sin\theta\int_{0}^{2\pi}{\rm d}\phi\frac{\partial f_{a}}{\partial v_{r}}(r,\boldsymbol{v}-\boldsymbol{v'})
\]
\[
\frac{\partial h}{\partial v_{t}}(r,v_{r},v_{t})=\int_{0}^{\infty}{\rm d}v'v'\int_{0}^{\pi}{\rm d}\theta\sin\theta\int_{0}^{2\pi}{\rm d}\phi\frac{\partial f_{a}}{\partial v_{t}}(r,\boldsymbol{v}-\boldsymbol{v'})
\]
where (at fixed $r$)
\[
\frac{\partial f_{a}}{\partial v_{r}}=\frac{\partial E}{\partial v_{r}}\frac{\partial f}{\partial E}+\frac{\partial L}{\partial v_{r}}\frac{\partial f}{\partial L}=-v_{r}\frac{\partial f}{\partial E}
\]
\[
\frac{\partial f_{a}}{\partial v_{t}}=\frac{\partial E}{\partial v_{t}}\frac{\partial f}{\partial E}+\frac{\partial L}{\partial v_{t}}\frac{\partial f}{\partial L}=-v_{t}\frac{\partial f}{\partial E}+r\frac{\partial f}{\partial L}
\]
 More generaly, consider the partial derivatives of $f_{E}(E,L)$.
Note that 
\[
f_{E}(E,L)=0\forall E,L<0
\]
hence its partial derivatives also vanish for $E,L<0$. Do the same
analysis as before to bound the integral. 

~

~
\begin{align*}
\frac{\partial^{2}g}{\partial v_{r}^{2}},\frac{\partial^{2}g}{\partial v_{t}\partial v_{r}},\frac{\partial^{2}g}{\partial v_{t}^{2}}
\end{align*}
\[
\frac{\partial^{2}f_{a}}{\partial v_{r}^{2}}=\frac{\partial}{\partial v_{r}}\left(-v_{r}\frac{\partial f}{\partial E}\right)=-\frac{\partial f}{\partial E}-v_{r}\left(\frac{\partial E}{\partial v_{r}}\frac{\partial^{2}f}{\partial E^{2}}+\frac{\partial L}{\partial v_{r}}\frac{\partial^{2}f}{\partial E\partial L}\right)=-\frac{\partial f}{\partial E}+v_{r}^{2}\frac{\partial^{2}f}{\partial E^{2}}
\]
\begin{align*}
\frac{\partial^{2}f_{a}}{\partial v_{t}^{2}} & =\frac{\partial}{\partial v_{t}}\left(-v_{t}\frac{\partial f}{\partial E}+r\frac{\partial f}{\partial L}\right)=-\frac{\partial f}{\partial E}-v_{t}\left(\frac{\partial E}{\partial v_{t}}\frac{\partial^{2}f}{\partial E^{2}}+\frac{\partial L}{\partial v_{t}}\frac{\partial^{2}f}{\partial L\partial E}\right)+r\left(\frac{\partial L}{\partial v_{t}}\frac{\partial f}{\partial E\partial L}+\frac{\partial E}{\partial v_{t}}\frac{\partial f}{\partial E^{2}}\right)\\
 & =-\frac{\partial f}{\partial E}+v_{t}^{2}\frac{\partial^{2}f}{\partial E^{2}}-rv_{t}\frac{\partial^{2}f}{\partial L\partial E}+r^{2}\frac{\partial f}{\partial E\partial L}-rv_{t}\frac{\partial f}{\partial E^{2}}
\end{align*}
\[
\frac{\partial^{2}f_{a}}{\partial v_{t}\partial v_{r}}=\frac{\partial}{\partial v_{t}}\left(-v_{r}\frac{\partial f}{\partial E}\right)=-v_{r}\left(\frac{\partial E}{\partial v_{t}}\frac{\partial^{2}f}{\partial E^{2}}+\frac{\partial L}{\partial v_{t}}\frac{\partial^{2}f}{\partial L\partial E}\right)=v_{r}v_{t}\frac{\partial^{2}f}{\partial E^{2}}-rv_{r}\frac{\partial^{2}f}{\partial L\partial E}
\]
\begin{align*}
\frac{\partial^{2}f_{a}}{\partial v_{r}\partial v_{t}} & =\frac{\partial}{\partial v_{r}}\left(-v_{t}\frac{\partial f}{\partial E}+r\frac{\partial f}{\partial L}\right)=-v_{t}\left(\frac{\partial E}{\partial v_{r}}\frac{\partial^{2}f}{\partial E^{2}}+\frac{\partial L}{\partial v_{r}}\frac{\partial^{2}f}{\partial E\partial L}\right)+r\left(\frac{\partial E}{\partial v_{r}}\frac{\partial^{2}f}{\partial E\partial L}+\frac{\partial L}{\partial v_{r}}\frac{\partial^{2}f}{\partial L^{2}}\right)\\
 & =v_{t}v_{r}\frac{\partial^{2}f}{\partial E^{2}}-rv_{r}\frac{\partial^{2}f}{\partial E\partial L}
\end{align*}
OK

\subsection{Recap}

\[
\frac{\partial f_{a}}{\partial v_{r}}=-v_{r}\frac{\partial f}{\partial E}
\]
\[
\frac{\partial f_{a}}{\partial v_{t}}=-v_{t}\frac{\partial f}{\partial E}+r\frac{\partial f}{\partial L}
\]
\[
\frac{\partial^{2}f_{a}}{\partial v_{r}^{2}}=-\frac{\partial f}{\partial E}+v_{r}^{2}\frac{\partial^{2}f}{\partial E^{2}}
\]
\begin{align*}
\frac{\partial^{2}f_{a}}{\partial v_{t}^{2}} & =-\frac{\partial f}{\partial E}+v_{t}^{2}\frac{\partial^{2}f}{\partial E^{2}}-rv_{t}\frac{\partial^{2}f}{\partial L\partial E}+r^{2}\frac{\partial f}{\partial E\partial L}-rv_{t}\frac{\partial f}{\partial E^{2}}
\end{align*}


\subsection{Angular and radial velocities}

The radial component is quite straightforward since 
\[
v_{ar}=(\boldsymbol{v}-\boldsymbol{v'})_{r}=v_{r}-v_{r}'=v_{r}-v'\cos\theta.
\]

On the other hand, the tangential component is a bit more tricky.
Let $\boldsymbol{v}=(v,\theta_{0},0)$ , where $v_{r}=v\cos\theta_{0}$
and $v_{t}=v\sin\theta_{0}$, and let $\boldsymbol{v'}=(v',\theta,\phi)$.
In cartesian coordinates, setting $(Ox)=\boldsymbol{v_{t}}$, $(Oz)=\boldsymbol{v_{r}}$
and $(Oy)$ such that $(Oxyz)$ is a direction orthonormal coordinate
system, we have
\[
v_{x}=v_{t}
\]
\[
v_{y}=0
\]
\[
v_{z}=v_{r}
\]
and
\[
v'_{x}=v'\sin\theta\cos\phi
\]
\[
v'_{y}=v'\sin\theta\sin\phi
\]
\[
v'_{z}=v'\cos\theta
\]
Then 
\[
v_{ax}=v_{t}-v'\sin\theta\cos\phi
\]
\[
v_{ay}=-v'\sin\theta\sin\phi
\]
\[
v_{az}=v_{r}-v'\cos\theta
\]
Therefore 
\[
v_{at}^{2}=v_{ax}^{2}+v_{ay}^{2}=(v_{t}-v'\sin\theta\cos\phi)^{2}+(v'\sin\theta\sin\phi)^{2}
\]
\[
v_{at}^{2}=v_{t}^{2}+v'^{2}\sin^{2}\theta\cos^{2}\phi-2v_{r}v'\sin\theta\cos\phi+v'^{2}\sin^{2}\theta\sin^{2}\phi
\]
\[
v_{at}^{2}=v_{t}^{2}+v'^{2}\sin^{2}\theta-2v_{t}v'\sin\theta\cos\phi
\]
The complete norm of $\boldsymbol{v_{a}}$ is 
\[
v_{a}^{2}=v_{ar}^{2}+v_{at}^{2}=(v_{r}-v'\cos\theta)^{2}+v_{t}^{2}+v'^{2}\sin^{2}\theta-2v_{t}v'\sin\theta\cos\phi
\]
\[
v_{a}^{2}=v_{r}^{2}+v'^{2}\cos^{2}\theta-2v_{r}v'\cos\theta+v_{t}^{2}+v'^{2}\sin^{2}\theta-2v_{t}v'\sin\theta\cos\phi
\]
\[
v_{a}^{2}=v^{2}+v'^{2}-2v'(v_{r}\cos\theta+v_{t}\sin\theta\cos\phi)
\]
which gives the binding energy per unit mass 
\[
E(r,v',\theta,\phi)=\psi(r)-\frac{1}{2}\left[v^{2}+v'^{2}-2v'(v_{r}\cos\theta+v_{t}\sin\theta\cos\phi)\right]
\]
and the angular momentum per unit mass
\[
L(r,v',\theta,\phi)=r(v_{t}^{2}+v'^{2}\sin^{2}\theta-2v_{t}v'\sin\theta\cos\phi)^{1/2}
\]


\subsection{Partial derivative of Fq(Ea,La)}

\[
E_{a}(r,v',\theta,\phi)=\psi(r)-\frac{1}{2}\left[v^{2}+v'^{2}-2v'(v_{r}\cos\theta+v_{t}\sin\theta\cos\phi)\right]
\]
\[
L_{a}(r,v',\theta,\phi)=r(v_{t}^{2}+v'^{2}\sin^{2}\theta-2v_{t}v'\sin\theta\cos\phi)^{1/2}
\]
\[
\frac{\partial E_{a}}{\partial v_{r}}=-v_{r}+v'\cos\theta
\]

\[
\frac{\partial E_{a}}{\partial v_{t}}=-v_{t}+v'\sin\theta\cos\phi
\]
\[
\frac{\partial L_{a}}{\partial v_{r}}=0
\]
\[
L_{a}^{2}=r(v_{t}^{2}+v'^{2}\sin^{2}\theta-2v_{t}v'\sin\theta\cos\phi)
\]
\[
2L_{a}\frac{\partial L_{a}}{\partial v_{t}}=2rv_{t}-2rv'\sin\theta\cos\phi\Rightarrow\frac{\partial L_{a}}{\partial v_{t}}=\frac{r}{L_{a}}\left(v_{t}-v'\sin\theta\cos\phi\right)
\]

~

First order:

\[
\frac{\partial}{\partial v_{r}}F(E_{a},L_{a})=\frac{\partial E_{a}}{\partial v_{r}}\frac{\partial F}{\partial E}+\frac{\partial L_{a}}{\partial v_{r}}\frac{\partial F}{\partial L}=\left(-v_{r}+v'\cos\theta\right)\frac{\partial F}{\partial E}
\]

~
\[
\frac{\partial}{\partial v_{t}}F(E_{a},L_{a})=\frac{\partial E_{a}}{\partial v_{t}}\frac{\partial F}{\partial E}+\frac{\partial L_{a}}{\partial v_{t}}\frac{\partial F}{\partial L}=\left(-v_{t}+v'\sin\theta\cos\phi\right)\frac{\partial F}{\partial E}+\frac{r}{L_{a}}\left(v_{t}-v'\sin\theta\cos\phi\right)\frac{\partial F}{\partial L}
\]

~
\begin{align*}
\frac{\partial^{2}}{\partial v_{r}^{2}}F(E_{a},L_{a}) & =\frac{\partial}{\partial v_{r}}\left[\left(-v_{r}+v'\cos\theta\right)\frac{\partial F}{\partial E}\right]=\frac{\partial}{\partial v_{r}}\left[\left(-v_{r}+v'\cos\theta\right)\right]\frac{\partial F}{\partial E}+\left(-v_{r}+v'\cos\theta\right)\frac{\partial}{\partial v_{r}}\left[\frac{\partial F}{\partial E}\right]\\
 & =-\frac{\partial F}{\partial E}+\left(-v_{r}+v'\cos\theta\right)\left(\frac{\partial E_{a}}{\partial v_{r}}\frac{\partial^{2}F}{\partial E{}^{2}}+\frac{\partial L_{a}}{\partial v_{r}}\frac{\partial F}{\partial L\partial E}\right)=-\frac{\partial F}{\partial E}+\left(-v_{r}+v'\cos\theta\right)^{2}\frac{\partial^{2}F}{\partial E{}^{2}}
\end{align*}

~

\begin{align*}
\frac{\partial^{2}}{\partial v_{t}\partial v_{r}}\left[f_{a}(r,\boldsymbol{v}-\boldsymbol{v'})\right] & =\frac{\partial}{\partial v_{t}}\left[\left(-v_{r}+v'\cos\theta\right)\frac{\partial F}{\partial E}\right]=\left(-v_{r}+v'\cos\theta\right)\frac{\partial}{\partial v_{t}}\left[\frac{\partial F}{\partial E}\right]\\
 & =\left(-v_{r}+v'\cos\theta\right)\left(\frac{\partial E_{a}}{\partial v_{t}}\frac{\partial^{2}F}{\partial E{}^{2}}+\frac{\partial L_{a}}{\partial v_{t}}\frac{\partial^{2}F}{\partial L\partial E}\right)\\
 & =\left(-v_{r}+v'\cos\theta\right)\left[\left(-v_{t}+v'\sin\theta\cos\phi\right)\frac{\partial^{2}F}{\partial E{}^{2}}+\frac{r}{L_{a}}\left(v_{t}-v'\sin\theta\cos\phi\right)\frac{\partial^{2}F}{\partial L\partial E}\right]
\end{align*}

~
\begin{align*}
\frac{\partial^{2}}{\partial v_{t}^{2}}F(E_{a},L_{a}) & =\frac{\partial}{\partial v_{t}}\left[\left(-v_{t}+v'\sin\theta\cos\phi\right)\frac{\partial F}{\partial E}+\frac{r}{L_{a}}\left(v_{t}-v'\sin\theta\cos\phi\right)\frac{\partial F}{\partial L}\right]\\
 & =-\frac{\partial F}{\partial E}+\left(-v_{t}+v'\sin\theta\cos\phi\right)\frac{\partial}{\partial v_{t}}\frac{\partial F}{\partial E}+\frac{\partial}{\partial v_{t}}\left[\frac{r}{L_{a}}\left(v_{t}-v'\sin\theta\cos\phi\right)\frac{\partial F}{\partial L}\right]
\end{align*}
with
\[
\frac{\partial}{\partial v_{t}}\frac{\partial F}{\partial E}=\left(-v_{t}+v'\sin\theta\cos\phi\right)\frac{\partial^{2}F}{\partial E{}^{2}}+\frac{r}{L_{a}}\left(v_{t}-v'\sin\theta\cos\phi\right)\frac{\partial^{2}F}{\partial L\partial E}
\]
\[
\frac{\partial}{\partial v_{t}}\left[\frac{r}{L_{a}}\left(v_{t}-v'\sin\theta\cos\phi\right)\frac{\partial F}{\partial L}\right]=\frac{\partial}{\partial v_{t}}\left[\frac{r}{L_{a}}\left(v_{t}-v'\sin\theta\cos\phi\right)\right]\frac{\partial F}{\partial L}+\frac{r}{L_{a}}\left(v_{t}-v'\sin\theta\cos\phi\right)\frac{\partial}{\partial v_{t}}\left[\frac{\partial F}{\partial L}\right]
\]
\begin{align*}
\frac{\partial}{\partial v_{t}}\left[\frac{r}{L_{a}}\left(v_{t}-v'\sin\theta\cos\phi\right)\right] & =\frac{rL_{a}-r\left(v_{t}-v'\sin\theta\cos\phi\right)\frac{\partial L_{a}}{\partial v_{t}}}{L_{a}^{2}}=\frac{rL_{a}-r\left(v_{t}-v'\sin\theta\cos\phi\right)\frac{r}{L_{a}}\left(v_{t}-v'\sin\theta\cos\phi\right)}{L_{a}^{2}}\\
 & =\frac{rL_{a}-\frac{r^{2}}{L_{a}}\left(v_{t}-v'\sin\theta\cos\phi\right)^{2}}{L_{a}^{2}}
\end{align*}
\begin{align*}
\frac{\partial}{\partial v_{t}}\left[\frac{\partial F}{\partial L}\right] & =\frac{\partial E_{a}}{\partial v_{t}}\frac{\partial^{2}F}{\partial E\partial L}+\frac{\partial L_{a}}{\partial v_{t}}\frac{\partial^{2}F}{\partial L{}^{2}}=\left(-v_{t}+v'\sin\theta\cos\phi\right)\frac{\partial^{2}F}{\partial E\partial L}+\frac{r}{L_{a}}\left(v_{t}-v'\sin\theta\cos\phi\right)\frac{\partial^{2}F}{\partial L{}^{2}}\\
 & =\left(-v_{t}+v'\sin\theta\cos\phi\right)\left(\frac{\partial^{2}F}{\partial E\partial L}-\frac{r}{L_{a}}\frac{\partial^{2}F}{\partial L{}^{2}}\right)
\end{align*}
thus
\begin{align*}
\frac{\partial^{2}}{\partial v_{t}^{2}}F(E_{a},L_{a}) & =-\frac{\partial F}{\partial E}+\left(-v_{t}+v'\sin\theta\cos\phi\right)^{2}\left(\frac{\partial^{2}F}{\partial E{}^{2}}-\frac{r}{L_{a}}\frac{\partial^{2}F}{\partial L\partial E}\right)+\frac{\partial}{\partial v_{t}}\left[\frac{r}{L_{a}}\left(v_{t}-v'\sin\theta\cos\phi\right)\frac{\partial F}{\partial L}\right]\\
 & =-\frac{\partial F}{\partial E}+\left(-v_{t}+v'\sin\theta\cos\phi\right)^{2}\left(\frac{\partial^{2}F}{\partial E{}^{2}}-\frac{r}{L_{a}}\frac{\partial^{2}F}{\partial L\partial E}\right)+\frac{rL_{a}-\frac{r^{2}}{L_{a}}\left(v_{t}-v'\sin\theta\cos\phi\right)^{2}}{L_{a}^{2}}\frac{\partial F}{\partial L}\\
 & +\frac{r}{L_{a}}\left(v_{t}-v'\sin\theta\cos\phi\right)\left(-v_{t}+v'\sin\theta\cos\phi\right)\left(\frac{\partial^{2}F}{\partial E\partial L}-\frac{r}{L_{a}}\frac{\partial^{2}F}{\partial L{}^{2}}\right)\\
 & =-\frac{\partial F}{\partial E'}+\frac{r}{L'}\frac{\partial F}{\partial L'}+\left(-v_{t}+v'\sin\theta\cos\phi\right)^{2}\left(\frac{\partial^{2}F}{\partial E'^{2}}-\frac{r}{L'}\frac{\partial^{2}F}{\partial L'\partial E'}-\frac{r^{2}}{L'^{3}}\frac{\partial F}{\partial L'}-\frac{r}{L'}\frac{\partial^{2}F}{\partial E'\partial L'}+\frac{r^{2}}{L'^{2}}\frac{\partial^{2}F}{\partial L'^{2}}\right)\\
 & =-\frac{\partial F}{\partial E'}+\frac{r}{L'}\frac{\partial F}{\partial L'}+\left(-v_{t}+v'\sin\theta\cos\phi\right)^{2}\left(\frac{\partial^{2}F}{\partial E'^{2}}-\frac{2r}{L'}\frac{\partial^{2}F}{\partial L'\partial E'}-\frac{r^{2}}{L'^{3}}\frac{\partial F}{\partial L'}+\frac{r^{2}}{L'^{2}}\frac{\partial^{2}F}{\partial L'^{2}}\right)
\end{align*}

~

Now compute the partial derivatives of $F(E,L)$
\[
F_{q}(E,L)=\frac{3\Gamma(6-q)}{2(2\pi)^{5/2}\Gamma(q/2)}E^{7/2-q}\mathbb{H}(0,\frac{q}{2},\frac{9}{2}-q,1;\frac{L^{2}}{2E})
\]
\[
\mathbb{H}(a,b,c,d;x)=\begin{cases}
\frac{\Gamma(a+b)}{\Gamma(c-a)\Gamma(a+d)}x^{a}._{2}F_{1}(a+b,1+a-c,a+d;x) & x\leq1\\
\frac{\Gamma(a+b)}{\Gamma(d-b)\Gamma(b+c)}x^{-b}._{2}F_{1}(a+b,1+b-d,b+c;\frac{1}{x}) & x\geq1
\end{cases}
\]

\[
\frac{\partial F}{\partial E}(E,L)=\frac{3\Gamma(6-q)\left(\frac{7}{2}-q\right)}{2(2\pi)^{5/2}\Gamma(q/2)}E^{5/2-q}\mathbb{H}(0,\frac{q}{2},\frac{9}{2}-q,1;\frac{L^{2}}{2E})-\frac{3\Gamma(6-q)}{2(2\pi)^{5/2}\Gamma(q/2)}E^{7/2-q}\frac{L^{2}}{2E^{2}}\frac{\partial\mathbb{H}}{\partial x}(0,\frac{q}{2},\frac{9}{2}-q,1;\frac{L^{2}}{2E})
\]
\[
\frac{\partial F}{\partial E}(E,L)=\frac{3\Gamma(6-q)}{2(2\pi)^{5/2}\Gamma(q/2)}E^{5/2-q}\left(\frac{7}{2}-q\right)\mathbb{H}(0,\frac{q}{2},\frac{9}{2}-q,1;\frac{L^{2}}{2E})-\frac{3\Gamma(6-q)}{2(2\pi)^{5/2}\Gamma(q/2)}E^{3/2-q}\frac{L^{2}}{2}\frac{\partial\mathbb{H}}{\partial x}(0,\frac{q}{2},\frac{9}{2}-q,1;\frac{L^{2}}{2E})
\]
thus
\[
\frac{\partial F}{\partial E}(E,L)=\frac{3\Gamma(6-q)E^{3/2-q}}{2(2\pi)^{5/2}\Gamma(q/2)}\left[E\left(\frac{7}{2}-q\right)\mathbb{H}(0,\frac{q}{2},\frac{9}{2}-q,1;\frac{L^{2}}{2E})-\frac{L^{2}}{2}\frac{\partial\mathbb{H}}{\partial x}(0,\frac{q}{2},\frac{9}{2}-q,1;\frac{L^{2}}{2E})\right]
\]
and, for $x<1$
\begin{align*}
\frac{\partial\mathbb{H}}{\partial x} & =\frac{\Gamma(a+b)a}{\Gamma(c-a)\Gamma(a+d)}x^{a-1}._{2}F_{1}(a+b,1+a-c,a+d;x)\\
 & +\frac{\Gamma(a+b)}{\Gamma(c-a)\Gamma(a+d)}x^{a}\frac{\partial._{2}F_{1}}{\partial x}(a+b,1+a-c,a+d;x)
\end{align*}
\begin{align*}
\frac{\partial\mathbb{H}}{\partial x} & =\frac{\Gamma(a+b)x^{a-1}}{\Gamma(c-a)\Gamma(a+d)}\bigg[a._{2}F_{1}(a+b,1+a-c,a+d;x)\\
 & +\frac{(a+b)(1+a-c)}{a+d}x._{2}F_{1}(a+b+1,a-c+2,a+d+1;x)\bigg]
\end{align*}
for $x>1$
\begin{align*}
\frac{\partial\mathbb{H}}{\partial x} & =\frac{\Gamma(a+b)}{\Gamma(d-b)\Gamma(b+c)}(-b)x^{-b-1}._{2}F_{1}(a+b,1+b-d,b+c;\frac{1}{x})\\
 & -\frac{\Gamma(a+b)}{\Gamma(d-b)\Gamma(b+c)}x^{-b}\frac{1}{x^{2}}\frac{\partial._{2}F_{1}}{\partial x}(a+b,1+b-d,b+c;\frac{1}{x})
\end{align*}
\begin{align*}
\frac{\partial\mathbb{H}}{\partial x} & =\frac{\Gamma(a+b)x^{-b-1}}{\Gamma(d-b)\Gamma(b+c)}\bigg[(-b)._{2}F_{1}(a+b,1+b-d,b+c;\frac{1}{x})\\
 & -\frac{(a+b)(1+b-d)}{b+c}x^{-1}._{2}F_{1}(a+b+1,b-d+2,b+c+1;\frac{1}{x})\bigg]
\end{align*}

~

\[
\mathbb{H}(a,b,c,d;x)=\begin{cases}
\frac{\Gamma(a+b)}{\Gamma(c-a)\Gamma(a+d)}x^{a}._{2}F_{1}(a+b,1+a-c,a+d;x) & x\leq1\\
\frac{\Gamma(a+b)}{\Gamma(d-b)\Gamma(b+c)}x^{-b}._{2}F_{1}(a+b,1+b-d,b+c;\frac{1}{x}) & x\geq1
\end{cases}
\]

x<1
\begin{align*}
\frac{\partial\mathbb{H}}{\partial x} & =\frac{\Gamma(a+b)x^{a-1}}{\Gamma(c-a)\Gamma(a+d)}\bigg[a._{2}F_{1}(a+b,1+a-c,a+d;x)\\
 & +\frac{(a+b)(1+a-c)}{a+d}x._{2}F_{1}(a+b+1,a-c+2,a+d+1;x)\bigg]
\end{align*}
x>1
\begin{align*}
\frac{\partial\mathbb{H}}{\partial x} & =\frac{\Gamma(a+b)x^{-b-1}}{\Gamma(d-b)\Gamma(b+c)}\bigg[(-b)._{2}F_{1}(a+b,1+b-d,b+c;\frac{1}{x})\\
 & -\frac{(a+b)(1+b-d)}{b+c}x^{-1}._{2}F_{1}(a+b+1,b-d+2,b+c+1;\frac{1}{x})\bigg]
\end{align*}

~

\[
\frac{\partial F_{q}}{\partial L}(E,L)=\frac{3\Gamma(6-q)}{2(2\pi)^{5/2}\Gamma(q/2)}E^{5/2-q}L\frac{\partial\mathbb{H}}{\partial x}(0,\frac{q}{2},\frac{9}{2}-q,1;\frac{L^{2}}{2E})
\]

~

\[
\frac{\partial F}{\partial E}(E,L)=\frac{3\Gamma(6-q)E^{3/2-q}}{2(2\pi)^{5/2}\Gamma(q/2)}\left[E\left(\frac{7}{2}-q\right)\mathbb{H}(0,\frac{q}{2},\frac{9}{2}-q,1;\frac{L^{2}}{2E})-\frac{L^{2}}{2}\frac{\partial\mathbb{H}}{\partial x}(0,\frac{q}{2},\frac{9}{2}-q,1;\frac{L^{2}}{2E})\right]
\]
\begin{align*}
\frac{\partial^{2}F}{\partial E^{2}}(E,L) & =\frac{3\Gamma(6-q)\left(\frac{3}{2}-q\right)E^{1/2-q}}{2(2\pi)^{5/2}\Gamma(q/2)}\left[E\left(\frac{7}{2}-q\right)\mathbb{H}(0,\frac{q}{2},\frac{9}{2}-q,1;\frac{L^{2}}{2E})-\frac{L^{2}}{2}\frac{\partial\mathbb{H}}{\partial x}(0,\frac{q}{2},\frac{9}{2}-q,1;\frac{L^{2}}{2E})\right]\\
 & +\frac{3\Gamma(6-q)E^{3/2-q}}{2(2\pi)^{5/2}\Gamma(q/2)}\left[\left(\frac{7}{2}-q\right)\mathbb{H}(0,\frac{q}{2},\frac{9}{2}-q,1;\frac{L^{2}}{2E})+E\left(\frac{7}{2}-q\right)\left(-\frac{L^{2}}{2E^{2}}\right)\frac{\partial\mathbb{H}}{\partial x}(0,\frac{q}{2},\frac{9}{2}-q,1;\frac{L^{2}}{2E})\right]\\
 & +\frac{3\Gamma(6-q)E^{3/2-q}}{2(2\pi)^{5/2}\Gamma(q/2)}\left[-\frac{L^{2}}{2}\left(-\frac{L^{2}}{2E^{2}}\right)\frac{\partial^{2}\mathbb{H}}{\partial x^{2}}(0,\frac{q}{2},\frac{9}{2}-q,1;\frac{L^{2}}{2E})\right]
\end{align*}
\begin{align*}
\frac{\partial^{2}F}{\partial E^{2}}(E,L) & =\frac{3\Gamma(6-q)E^{3/2-q}}{2(2\pi)^{5/2}\Gamma(q/2)}\left[\left(\frac{7}{2}-q\right)\left(\frac{3}{2}-q\right)\mathbb{H}(0,\frac{q}{2},\frac{9}{2}-q,1;\frac{L^{2}}{2E})-\left(\frac{3}{2}-q\right)\frac{L^{2}}{2E}\frac{\partial\mathbb{H}}{\partial x}(0,\frac{q}{2},\frac{9}{2}-q,1;\frac{L^{2}}{2E})\right]\\
 & +\frac{3\Gamma(6-q)E^{3/2-q}}{2(2\pi)^{5/2}\Gamma(q/2)}\left[\left(\frac{7}{2}-q\right)\mathbb{H}(0,\frac{q}{2},\frac{9}{2}-q,1;\frac{L^{2}}{2E})-\left(\frac{7}{2}-q\right)\frac{L^{2}}{2E}\frac{\partial\mathbb{H}}{\partial x}(0,\frac{q}{2},\frac{9}{2}-q,1;\frac{L^{2}}{2E})\right]\\
 & +\frac{3\Gamma(6-q)E^{3/2-q}}{2(2\pi)^{5/2}\Gamma(q/2)}\left[\frac{L^{4}}{4E^{2}}\frac{\partial^{2}\mathbb{H}}{\partial x^{2}}(0,\frac{q}{2},\frac{9}{2}-q,1;\frac{L^{2}}{2E})\right]
\end{align*}

\begin{align*}
\frac{\partial^{2}F}{\partial E^{2}}(E,L) & =\frac{3\Gamma(6-q)E^{3/2-q}}{2(2\pi)^{5/2}\Gamma(q/2)}\bigg[\left(\frac{7}{2}-q\right)\left(\frac{3}{2}-q\right)\mathbb{H}(0,\frac{q}{2},\frac{9}{2}-q,1;\frac{L^{2}}{2E})-\left(\frac{3}{2}-q\right)\frac{L^{2}}{2E}\frac{\partial\mathbb{H}}{\partial x}(0,\frac{q}{2},\frac{9}{2}-q,1;\frac{L^{2}}{2E})\\
 & +\left(\frac{7}{2}-q\right)\mathbb{H}(0,\frac{q}{2},\frac{9}{2}-q,1;\frac{L^{2}}{2E})-\left(\frac{7}{2}-q\right)\frac{L^{2}}{2E}\frac{\partial\mathbb{H}}{\partial x}(0,\frac{q}{2},\frac{9}{2}-q,1;\frac{L^{2}}{2E})+\frac{L^{4}}{4E^{2}}\frac{\partial^{2}\mathbb{H}}{\partial x^{2}}(0,\frac{q}{2},\frac{9}{2}-q,1;\frac{L^{2}}{2E})\bigg]
\end{align*}
donc
\begin{align*}
\frac{\partial^{2}F}{\partial E^{2}}(E,L) & =\frac{3\Gamma(6-q)E^{3/2-q}}{2(2\pi)^{5/2}\Gamma(q/2)}\bigg[\left(\frac{7}{2}-q\right)\left(\frac{5}{2}-q\right)\mathbb{H}(0,\frac{q}{2},\frac{9}{2}-q,1;\frac{L^{2}}{2E})\\
 & -\left(5-2q\right)\frac{L^{2}}{2E}\frac{\partial\mathbb{H}}{\partial x}(0,\frac{q}{2},\frac{9}{2}-q,1;\frac{L^{2}}{2E})+\frac{L^{4}}{4E^{2}}\frac{\partial^{2}\mathbb{H}}{\partial x^{2}}(0,\frac{q}{2},\frac{9}{2}-q,1;\frac{L^{2}}{2E})\bigg]
\end{align*}
x<1
\begin{align*}
\frac{\partial\mathbb{H}}{\partial x} & =\frac{\Gamma(a+b)x^{a-1}}{\Gamma(c-a)\Gamma(a+d)}\bigg[a._{2}F_{1}(a+b,1+a-c,a+d;x)\\
 & +\frac{(a+b)(1+a-c)}{a+d}x._{2}F_{1}(a+b+1,a-c+2,a+d+1;x)\bigg]
\end{align*}
\begin{align*}
\frac{\partial^{2}\mathbb{H}}{\partial x^{2}} & =\frac{\Gamma(a+b)(a-1)x^{a-2}}{\Gamma(c-a)\Gamma(a+d)}\bigg[a._{2}F_{1}(a+b,1+a-c,a+d;x)\\
 & +\frac{(a+b)(1+a-c)}{a+d}x._{2}F_{1}(a+b+1,a-c+2,a+d+1;x)\bigg]\\
 & +\frac{\Gamma(a+b)x^{a-1}}{\Gamma(c-a)\Gamma(a+d)}\bigg[a\frac{(a+b)(1+a-c)}{a+d}._{2}F_{1}(a+b+1,a-c+2,a+d+1;x)\\
 & +\frac{(a+b)(1+a-c)}{a+d}._{2}F_{1}(a+b+1,a-c+2,a+d+1;x)\\
 & +\frac{(a+b)(1+a-c)}{a+d}x\frac{(a+b+1)(a-c+2)}{a+d+1}._{2}F_{1}(a+b+2,a-c+3,a+d+2;x)\bigg]
\end{align*}
\begin{align*}
\frac{\partial^{2}\mathbb{H}}{\partial x^{2}} & =\frac{\Gamma(a+b)x^{a-2}}{\Gamma(c-a)\Gamma(a+d)}\bigg[a(a-1)._{2}F_{1}(a+b,1+a-c,a+d;x)\\
 & +\frac{(a-1)(a+b)(1+a-c)}{a+d}x._{2}F_{1}(a+b+1,a-c+2,a+d+1;x)\\
 & +\frac{a(a+b)(1+a-c)}{a+d}x._{2}F_{1}(a+b+1,a-c+2,a+d+1;x)\\
 & +\frac{(a+b)(1+a-c)}{a+d}x._{2}F_{1}(a+b+1,a-c+2,a+d+1;x)\\
 & +\frac{(a+b)(1+a-c)}{a+d}x^{2}\frac{(a+b+1)(a-c+2)}{a+d+1}._{2}F_{1}(a+b+2,a-c+3,a+d+2;x)\bigg]
\end{align*}
\begin{align*}
\frac{\partial^{2}\mathbb{H}}{\partial x^{2}} & =\frac{\Gamma(a+b)x^{a-2}}{\Gamma(c-a)\Gamma(a+d)}\bigg[a(a-1)._{2}F_{1}(a+b,1+a-c,a+d;x)\\
 & +\frac{2a(a+b)(1+a-c)}{a+d}x._{2}F_{1}(a+b+1,a-c+2,a+d+1;x)\\
 & +\frac{(a+b)(1+a-c)}{a+d}\frac{(a+b+1)(a-c+2)}{a+d+1}x^{2}._{2}F_{1}(a+b+2,a-c+3,a+d+2;x)\bigg]
\end{align*}
x>1
\begin{align*}
\frac{\partial\mathbb{H}}{\partial x} & =\frac{\Gamma(a+b)x^{-b-1}}{\Gamma(d-b)\Gamma(b+c)}\bigg[(-b)._{2}F_{1}(a+b,1+b-d,b+c;\frac{1}{x})\\
 & -\frac{(a+b)(1+b-d)}{b+c}x^{-1}._{2}F_{1}(a+b+1,b-d+2,b+c+1;\frac{1}{x})\bigg]
\end{align*}
\begin{align*}
\frac{\partial^{2}\mathbb{H}}{\partial x^{2}} & =\frac{\Gamma(a+b)(-b-1)x^{-b-2}}{\Gamma(d-b)\Gamma(b+c)}\bigg[(-b)._{2}F_{1}(a+b,1+b-d,b+c;\frac{1}{x})\\
 & -\frac{(a+b)(1+b-d)}{b+c}x^{-1}._{2}F_{1}(a+b+1,b-d+2,b+c+1;\frac{1}{x})\bigg]\\
 & +\frac{\Gamma(a+b)x^{-b-1}}{\Gamma(d-b)\Gamma(b+c)}\bigg[(-\frac{1}{x^{2}})\frac{(a+b)(1+b-d)}{b+c}(-b)._{2}F_{1}(a+b+1,b-d+2,b+c+1;\frac{1}{x})\\
 & -\frac{(a+b)(1+b-d)}{b+c}(-1)x^{-2}._{2}F_{1}(a+b+1,b-d+2,b+c+1;\frac{1}{x})\\
 & -\frac{(a+b)(1+b-d)}{b+c}x^{-1}(\frac{-1}{x^{2}})\frac{(a+b+1)(b-d+2)}{b+c+1}._{2}F_{1}(a+b+2,b-d+3,b+c+2;\frac{1}{x})\bigg]
\end{align*}
\begin{align*}
\frac{\partial^{2}\mathbb{H}}{\partial x^{2}} & =\frac{\Gamma(a+b)x^{-b-2}}{\Gamma(d-b)\Gamma(b+c)}\bigg[(-b-1)(-b)._{2}F_{1}(a+b,1+b-d,b+c;\frac{1}{x})\\
 & -\frac{(a+b)(1+b-d)}{b+c}(-b-1)x^{-1}._{2}F_{1}(a+b+1,b-d+2,b+c+1;\frac{1}{x})\\
 & +(-\frac{x}{x^{2}})\frac{(a+b)(1+b-d)}{b+c}(-b)._{2}F_{1}(a+b+1,b-d+2,b+c+1;\frac{1}{x})\\
 & -\frac{(a+b)(1+b-d)}{b+c}x(-1)x^{-2}._{2}F_{1}(a+b+1,b-d+2,b+c+1;\frac{1}{x})\\
 & -\frac{(a+b)(1+b-d)}{b+c}xx^{-1}(\frac{-1}{x^{2}})\frac{(a+b+1)(b-d+2)}{b+c+1}._{2}F_{1}(a+b+2,b-d+3,b+c+2;\frac{1}{x})\bigg]
\end{align*}
\begin{align*}
\frac{\partial^{2}\mathbb{H}}{\partial x^{2}} & =\frac{\Gamma(a+b)x^{-b-2}}{\Gamma(d-b)\Gamma(b+c)}\bigg[b(b+1)._{2}F_{1}(a+b,1+b-d,b+c;\frac{1}{x})\\
 & +\frac{(a+b)(1+b-d)}{b+c}(b+1)x^{-1}._{2}F_{1}(a+b+1,b-d+2,b+c+1;\frac{1}{x})\\
 & +\frac{(a+b)(1+b-d)}{b+c}(b)x^{-1}._{2}F_{1}(a+b+1,b-d+2,b+c+1;\frac{1}{x})\\
 & +\frac{(a+b)(1+b-d)}{b+c}x^{-1}._{2}F_{1}(a+b+1,b-d+2,b+c+1;\frac{1}{x})\\
 & +\frac{(a+b)(1+b-d)}{b+c}\frac{(a+b+1)(b-d+2)}{b+c+1}x^{-2}._{2}F_{1}(a+b+2,b-d+3,b+c+2;\frac{1}{x})\bigg]
\end{align*}
\begin{align*}
\frac{\partial^{2}\mathbb{H}}{\partial x^{2}} & =\frac{\Gamma(a+b)x^{-b-2}}{\Gamma(d-b)\Gamma(b+c)}\bigg[b(b+1)._{2}F_{1}(a+b,1+b-d,b+c;\frac{1}{x})\\
 & +\frac{(2b+2)(a+b)(1+b-d)}{b+c}x^{-1}._{2}F_{1}(a+b+1,b-d+2,b+c+1;\frac{1}{x})\\
 & +\frac{(a+b)(1+b-d)}{b+c}\frac{(a+b+1)(b-d+2)}{b+c+1}x^{-2}._{2}F_{1}(a+b+2,b-d+3,b+c+2;\frac{1}{x})\bigg]
\end{align*}
finalement

x<1
\begin{align*}
\frac{\partial^{2}\mathbb{H}}{\partial x^{2}} & =\frac{\Gamma(a+b)x^{a-2}}{\Gamma(c-a)\Gamma(a+d)}\bigg[a(a-1)._{2}F_{1}(a+b,1+a-c,a+d;x)\\
 & +\frac{2a(a+b)(1+a-c)}{a+d}x._{2}F_{1}(a+b+1,a-c+2,a+d+1;x)\\
 & +\frac{(a+b)(1+a-c)}{a+d}\frac{(a+b+1)(a-c+2)}{a+d+1}x^{2}._{2}F_{1}(a+b+2,a-c+3,a+d+2;x)\bigg]
\end{align*}
x>1
\begin{align*}
\frac{\partial^{2}\mathbb{H}}{\partial x^{2}} & =\frac{\Gamma(a+b)x^{-b-2}}{\Gamma(d-b)\Gamma(b+c)}\bigg[b(b+1)._{2}F_{1}(a+b,1+b-d,b+c;\frac{1}{x})\\
 & +\frac{(2b+2)(a+b)(1+b-d)}{b+c}x^{-1}._{2}F_{1}(a+b+1,b-d+2,b+c+1;\frac{1}{x})\\
 & +\frac{(a+b)(1+b-d)}{b+c}\frac{(a+b+1)(b-d+2)}{b+c+1}x^{-2}._{2}F_{1}(a+b+2,b-d+3,b+c+2;\frac{1}{x})\bigg]
\end{align*}

~

~

\begin{align*}
\frac{\partial^{2}F_{q}}{\partial E\partial L}(E,L) & =\frac{3\Gamma(6-q)}{2(2\pi)^{5/2}\Gamma(q/2)}\left(\frac{5}{2}-q\right)E^{3/2-q}L\frac{\partial\mathbb{H}}{\partial x}(0,\frac{q}{2},\frac{9}{2}-q,1;\frac{L^{2}}{2E})\\
 & +\frac{3\Gamma(6-q)}{2(2\pi)^{5/2}\Gamma(q/2)}E^{5/2-q}L\left(-\frac{L^{2}}{2E^{2}}\right)\frac{\partial^{2}\mathbb{H}}{\partial x^{2}}(0,\frac{q}{2},\frac{9}{2}-q,1;\frac{L^{2}}{2E})
\end{align*}
\begin{align*}
\frac{\partial^{2}F_{q}}{\partial E\partial L}(E,L) & =\frac{3\Gamma(6-q)LE^{1/2-q}}{2(2\pi)^{5/2}\Gamma(q/2)}\bigg[\left(\frac{5}{2}-q\right)E\frac{\partial\mathbb{H}}{\partial x}(0,\frac{q}{2},\frac{9}{2}-q,1;\frac{L^{2}}{2E})\\
 & -\frac{L^{2}}{2}\frac{\partial^{2}\mathbb{H}}{\partial x^{2}}(0,\frac{q}{2},\frac{9}{2}-q,1;\frac{L^{2}}{2E})\bigg]
\end{align*}

~

\[
\frac{\partial F_{q}}{\partial L}(E,L)=\frac{3\Gamma(6-q)}{2(2\pi)^{5/2}\Gamma(q/2)}E^{5/2-q}L\frac{\partial\mathbb{H}}{\partial x}(0,\frac{q}{2},\frac{9}{2}-q,1;\frac{L^{2}}{2E})
\]

\begin{align*}
\frac{\partial^{2}F_{q}}{\partial L^{2}}(E,L) & =\frac{3\Gamma(6-q)}{2(2\pi)^{5/2}\Gamma(q/2)}E^{5/2-q}\frac{\partial\mathbb{H}}{\partial x}(0,\frac{q}{2},\frac{9}{2}-q,1;\frac{L^{2}}{2E})\\
 & +\frac{3\Gamma(6-q)}{2(2\pi)^{5/2}\Gamma(q/2)}E^{3/2-q}L^{2}\frac{\partial^{2}\mathbb{H}}{\partial x^{2}}(0,\frac{q}{2},\frac{9}{2}-q,1;\frac{L^{2}}{2E})
\end{align*}
\begin{align*}
\frac{\partial^{2}F_{q}}{\partial L^{2}}(E,L) & =\frac{3\Gamma(6-q)E^{3/2-q}}{2(2\pi)^{5/2}\Gamma(q/2)}\left[E\frac{\partial\mathbb{H}}{\partial x}(0,\frac{q}{2},\frac{9}{2}-q,1;\frac{L^{2}}{2E})+L^{2}\frac{\partial^{2}\mathbb{H}}{\partial x^{2}}(0,\frac{q}{2},\frac{9}{2}-q,1;\frac{L^{2}}{2E})\right]
\end{align*}

Case $q=0$
\[
F_{0}(E)=\frac{3}{7\pi^{3}}(2E)^{7/2}
\]
\[
\frac{\partial F_{0}}{\partial E}(E)=\frac{3}{\pi^{3}}(2E)^{5/2}
\]
\[
\frac{\partial^{2}F_{0}}{\partial E^{2}}(E)=\frac{15}{\pi^{3}}(2E)^{3/2}
\]

Case $q=2$
\[
F_{2}(E,L)=\begin{cases}
\frac{6}{(2\pi)^{3}}(2E-L^{2})^{3/2} & L^{2}/(2E)\leq1\\
0 & L^{2}/(2E)\geq0
\end{cases}
\]
\[
\frac{\partial F_{2}}{\partial E}(E,L)=\frac{18}{(2\pi)^{3}}(2E-L^{2})^{1/2}
\]
\[
\frac{\partial F_{2}}{\partial L}(E,L)=\frac{-18L}{(2\pi)^{3}}(2E-L^{2})^{1/2}
\]
\[
\frac{\partial^{2}F_{2}}{\partial E^{2}}(E,L)=\frac{18}{(2\pi)^{3}}(2E-L^{2})^{-1/2}
\]
\[
\frac{\partial F_{2}}{\partial E\partial L}(E,L)=\frac{-18L}{(2\pi)^{3}}(2E-L^{2})^{-1/2}
\]
\[
\frac{\partial^{2}F_{2}}{\partial L^{2}}(E,L)=-\frac{18}{(2\pi)^{3}}(2E-L^{2})^{1/2}+\frac{18L^{2}}{(2\pi)^{3}}(2E-L^{2})^{-1/2}
\]


\subsection{$L_{{\rm circ}}(E)$}

We have , for any particle with binding energy $E$ and angular momentum
$L$:
\[
E=\psi(r)-\frac{L^{2}}{2r^{2}}-\frac{\dot{r}^{2}}{2}
\]
hence 
\[
\frac{L^{2}}{2r^{2}}=\psi(r)-E-\frac{\dot{r}^{2}}{2}\leq\psi(r)-E
\]
Hence for any $r$ 
\[
L^{2}\leq2r^{2}(\psi(r)-E)=z(r)
\]
Same study as for $E_{{\rm c}}(L)$. On the bounds of the orbit, $\dot{r}=0$
and : 
\[
L^{2}=2r^{2}(\psi(r)-E)=z(r;E)
\]
Two solution as already shown. One global maximum $R_{E}>0$. For
an orbit with $L_{{\rm c}}(E)=\sqrt{\max_{r\geq0}z(r;E)}$, it is
confined at $R_{E}$ where $v_{r}(R)=0$, hence is circular and higher
$L>L_{{\rm c}}(E)$ are forbidden for that particular energy $E$.

Finding the maximum of $z(r;E)$ is the same as solving 
\[
z'(r;E)=4r(\psi(r)-E)+2r^{2}\psi'(r)=0
\]
We have 
\[
z'(r;E)=4r\left(\frac{1}{\sqrt{1+r^{2}}}-E\right)-\frac{2r^{3}}{(1+r^{2})^{3/2}}
\]

For $\frac{1}{\sqrt{1+r_{E}^{2}}}=E\leftrightarrow r_{E}=\sqrt{E^{-2}-1}$,
we have 
\[
z'(r_{E};E)=-\frac{2r_{E}^{3}}{(1+r_{E}^{2})^{3/2}}\in]-2,0[
\]
By convexity, use Newton's method starting from $r_{0}^{E}=r_{E}$.
It gives a decreasing sequence. Therefore a good stopping condition
is to get the lowest $N$ such that $z'(r_{N}^{E}+\epsilon)>0$.

~

~$v>v_{a}$

\begin{align*}
\frac{1}{|v-v_{a}|} & =\frac{1}{\sqrt{v^{2}+v_{a}^{2}-2v_{a}v\cos\theta}}=\frac{1}{v\sqrt{1+\left(\frac{v_{a}}{v}\right)^{2}-2\frac{v_{a}}{v}\cos\theta}}\\
 & =\frac{1}{v}\sum_{\ell=0}^{\infty}P_{\ell}(\cos\theta)\left(\frac{v_{a}}{v}\right)^{\ell}
\end{align*}
\[
\frac{f(v_{a})}{|v-v_{a}|}=\frac{f(v_{a})}{v}\sum_{\ell=0}^{\infty}P_{\ell}(\cos\theta)\left(\frac{v_{a}}{v}\right)^{\ell}
\]
\[
\frac{\partial}{\partial v_{r}}\frac{f(v_{a})}{|v-v_{a}|}=f(v_{a})\sum_{\ell=0}^{\infty}P_{\ell}(\cos\theta)\frac{v_{a}^{\ell}}{v^{\ell+1}}
\]

~

~ https://en.wikipedia.org/wiki/Legendre\_polynomials (section recurrence
relations)

\[
\frac{x^{2}-1}{n}\frac{{\rm d}P_{n}}{{\rm d}x}(x)=xP_{n}(x)-P_{n-1}(x)
\]
hence
\[
\frac{{\rm d}P_{n}}{{\rm d}x}(x)=\frac{n}{x^{2}-1}\left[xP_{n}(x)-P_{n-1}(x)\right]
\]

\end{document}
